\documentclass[psamsfonts, 12pt]{amsart}
%
%-------Packages---------
%
\usepackage[h margin=1 in, v margin=1 in]{geometry}
\usepackage{amssymb,amsfonts}
\usepackage[all,arc]{xy}
\usepackage{tikz-cd}
\usepackage{accents}
\usepackage{enumerate}
\usepackage{mathrsfs}
\usepackage{amsthm}
\usepackage{mathpazo}
\usepackage[mathcal]{eucal}
\usepackage{float}
\usepackage[backend=biber]{biblatex}
\addbibresource{bibliography.bib}
%\usepackage{charter} %another font
%\usepackage{eulervm} %Vakil font
\usepackage{yfonts}
\usepackage{mathtools}
\usepackage{enumitem}
\usepackage{mathrsfs}
\usepackage{pifont}
\usepackage{fourier-orns}
\usepackage{adforn}
\usepackage[all]{xy}
\usepackage{hyperref}
\usepackage{url}
\usepackage{mathtools}
\usepackage{graphicx}
\usepackage{pdfsync}
\usepackage{mathdots}
\usepackage{calligra}
\usepackage{import}
\usepackage{xifthen}
\usepackage{pdfpages}
\usepackage{transparent}

\usepackage{tgpagella}
\usepackage[T1]{fontenc}
%
\usepackage{listings}
\usepackage{color}

\definecolor{dkgreen}{rgb}{0,0.6,0}
\definecolor{gray}{rgb}{0.5,0.5,0.5}
\definecolor{mauve}{rgb}{0.58,0,0.82}

\lstset{frame=tb,
  language=Matlab,
  aboveskip=3mm,
  belowskip=3mm,
  showstringspaces=false,
  columns=flexible,
  basicstyle={\small\ttfamily},
  numbers=none,
  numberstyle=\tiny\color{gray},
  keywordstyle=\color{blue},
  commentstyle=\color{dkgreen},
  stringstyle=\color{mauve},
  breaklines=true,
  breakatwhitespace=true,
  tabsize=3
  }
% Markers for end of environments
\newcommand{\exerQED}{\smash\adfhalfleftarrowhead}
\newcommand{\defnQED}{\ding{118}}
%
%--------Theorem Environments--------
%
\newtheorem{thm}{Theorem}[section]
\newtheorem*{thm*}{Theorem}
\newtheorem{cor}[thm]{Corollary}
\newtheorem{prop}[thm]{Proposition}
\newtheorem{lem}[thm]{Lemma}
\newtheorem*{lem*}{Lemma}
\newtheorem{conj}[thm]{Conjecture}
\newtheorem{quest}[thm]{Question}
%
\newenvironment{defn}
  {\pushQED{\qed}\renewcommand{\qedsymbol}{\defnQED}\defnx}
  {\popQED\enddefnx}
\newenvironment{exer}
  {\pushQED{\qed}\renewcommand{\qedsymbol}{\exerQED}\exerx}
  {\popQED\endexerx}
\theoremstyle{definition}
%\newtheorem{defn}[thm]{Definition}
\newtheorem{defnx}[thm]{Definition}
\newtheorem{con}[thm]{Construction}
\newtheorem{exmp}[thm]{Example}
\newtheorem{exmps}[thm]{Examples}
\newtheorem{notn}[thm]{Notation}
\newtheorem{notns}[thm]{Notations}
\newtheorem{addm}[thm]{Addendum}
\newtheorem{exerx}[thm]{Exercise}
%
\theoremstyle{remark}
\newtheorem{rem}[thm]{Remark}
\newtheorem*{claim}{Claim}
\newtheorem*{aside*}{Aside}
\newtheorem*{rem*}{Remark}
\newtheorem*{hint*}{Hint}
\newtheorem*{note}{Note}
\newtheorem{rems}[thm]{Remarks}
\newtheorem{warn}[thm]{Warning}
\newtheorem{sch}[thm]{Scholium}
%
%--------Macros--------
\renewcommand{\qedsymbol}{$\blacksquare$}
\renewcommand{\sl}{\mathfrak{sl}}
\newcommand{\Bord}{\mathsf{Bord}}
\renewcommand{\hom}{\mathrm{Hom}}
\renewcommand{\emptyset}{\varnothing}
\renewcommand{\O}{\mathcal{O}}
\newcommand{\R}{\mathbb{R}}
\newcommand{\ib}[1]{\textbf{\textit{#1}}}
\newcommand{\Q}{\mathbb{Q}}
\newcommand{\Z}{\mathbb{Z}}
\newcommand{\N}{\mathbb{N}}
\newcommand{\C}{\mathbb{C}}
\newcommand{\A}{\mathbb{A}}
\newcommand{\F}{\mathbb{F}}
\newcommand{\M}{\mathcal{M}}
\newcommand{\dbar}{\overline{\partial}}
\newcommand{\zbar}{\overline{z}}
\renewcommand{\S}{\mathbb{S}}
\newcommand{\V}{\vec{v}}
\newcommand{\RP}{\mathbb{RP}}
\newcommand{\CP}{\mathbb{CP}}
\newcommand{\B}{\mathcal{B}}
\newcommand{\GL}{\mathrm{GL}}
\newcommand{\SL}{\mathrm{SL}}
\newcommand{\SP}{\mathrm{SP}}
\newcommand{\SO}{\mathrm{SO}}
\newcommand{\SU}{\mathrm{SU}}
\newcommand{\gl}{\mathfrak{gl}}
\newcommand{\g}{\mathfrak{g}}
\newcommand{\Bun}{\mathsf{Bun}}
\newcommand{\inv}{^{-1}}
\newcommand{\bra}[2]{ \left[ #1, #2 \right] }
\newcommand{\set}[1]{\left\lbrace #1 \right\rbrace}
\newcommand{\abs}[1]{\left\lvert#1\right\rvert}
\newcommand{\norm}[1]{\left\lVert#1\right\rVert}
\newcommand{\transv}{\mathrel{\text{\tpitchfork}}}
\newcommand*{\dt}[1]{%
   \accentset{\mbox{\large\bfseries .}}{#1}}
\newcommand{\defeq}{\vcentcolon=}
\newcommand{\enumbreak}{\ \\ \vspace{-\baselineskip}}
\let\oldexists\exists
\renewcommand\exists{\oldexists~}
\let\oldL\L
\renewcommand\L{\mathfrak{L}}
\makeatletter
\newcommand{\incfig}[2]{%
    \fontsize{48pt}{50pt}\selectfont
    \def\svgwidth{\columnwidth}
    \scalebox{#2}{\input{#1.pdf_tex}}
}
%
\newcommand{\tpitchfork}{%
  \vbox{
    \baselineskip\z@skip
    \lineskip-.52ex
    \lineskiplimit\maxdimen
    \m@th
    \ialign{##\crcr\hidewidth\smash{$-$}\hidewidth\crcr$\pitchfork$\crcr}
  }%
}
\makeatother
\newcommand{\bd}{\partial}
\newcommand{\lang}{\begin{picture}(5,7)
\put(1.1,2.5){\rotatebox{45}{\line(1,0){6.0}}}
\put(1.1,2.5){\rotatebox{315}{\line(1,0){6.0}}}
\end{picture}}
\newcommand{\rang}{\begin{picture}(5,7)
\put(.1,2.5){\rotatebox{135}{\line(1,0){6.0}}}
\put(.1,2.5){\rotatebox{225}{\line(1,0){6.0}}}
\end{picture}}
\DeclareMathOperator{\id}{id}
\DeclareMathOperator{\im}{Im}
\DeclareMathOperator{\codim}{codim}
\DeclareMathOperator{\coker}{coker}
\DeclareMathOperator{\supp}{supp}
\DeclareMathOperator{\inter}{Int}
\DeclareMathOperator{\sign}{sign}
\DeclareMathOperator{\sgn}{sgn}
\DeclareMathOperator{\indx}{ind}
\DeclareMathOperator{\alt}{Alt}
\DeclareMathOperator{\Aut}{Aut}
\DeclareMathOperator{\trace}{trace}
\DeclareMathOperator{\ad}{ad}
\DeclareMathOperator{\End}{End}
\DeclareMathOperator{\Ad}{Ad}
\DeclareMathOperator{\Lie}{Lie}
\DeclareMathOperator{\spn}{span}
\DeclareMathOperator{\dv}{div}
\DeclareMathOperator{\grad}{grad}
\DeclareMathOperator{\Sym}{Sym}
\DeclareMathOperator{\sheafhom}{\mathscr{H}\text{\kern -3pt {\calligra\large om}}\,}
\newcommand*\myhrulefill{%
   \leavevmode\leaders\hrule depth-2pt height 2.4pt\hfill\kern0pt}
\newcommand\niceending[1]{%
  \begin{center}%
    \LARGE \myhrulefill \hspace{0.2cm} #1 \hspace{0.2cm} \myhrulefill%
  \end{center}}
\newcommand*\sectionend{\niceending{\decofourleft\decofourright}}
\newcommand*\subsectionend{\niceending{\decosix}}
\def\upint{\mathchoice%
    {\mkern13mu\overline{\vphantom{\intop}\mkern7mu}\mkern-20mu}%
    {\mkern7mu\overline{\vphantom{\intop}\mkern7mu}\mkern-14mu}%
    {\mkern7mu\overline{\vphantom{\intop}\mkern7mu}\mkern-14mu}%
    {\mkern7mu\overline{\vphantom{\intop}\mkern7mu}\mkern-14mu}%
  \int}
\def\lowint{\mkern3mu\underline{\vphantom{\intop}\mkern7mu}\mkern-10mu\int}
%
%--------Hypersetup--------
%
\hypersetup{
    colorlinks,
    citecolor=black,
    filecolor=black,
    linkcolor=blue,
}
%
%--------Solution--------
%
\newenvironment{solution}
  {\begin{proof}[Solution]}
  {\end{proof}}
%
%--------Graphics--------
%
%\graphicspath{ {images/} }
%
\begin{document}
%
\author{Jeffrey Jiang}
%
\title{Yang-Mills Minicourse Notes}
%
\maketitle
%
These are notes for a Summer 2020 minicourse following the paper
"The Yang-Mills Equations over Riemann Surfaces" \cite{AB} by Atiyah and Bott.
%
\tableofcontents
%
\section*{Notation and Conventions}
%
We use Einstein summation notation, i.e. indices that appear on the top and bottom
of an expression are implicitly summed over. For example
\[
a_idx^i = \sum_i a_idx^i
\]
For a smooth manifold $X$, we let $\Omega^k_X$ denote the space of differential
$k$-forms. When $X$ is a complex manifold, we let $\mathcal{A}^k_X$ denote the
space of smooth complex-valued $k$-forms, and $\mathcal{A}^{p,q}_X$ the
space of smooth $(p,q)$-forms. We reserve $\Omega^k_X$ and $\Omega^{p,q}_X$
for the spaces of holomorphic $k$ and $(p,q)$ forms respectively.
%
%
\section{Principal Bundles and Connections}
%
Good references for this material would be \cite{KobNom} and \cite{Tu} \\

Fix a close manifold $X$ (compact and without boundary) and a Lie group $G$.
%
\begin{defn}
A \ib{principal $G$-bundle} is a fiber bundle $\pi : P \to M$ with a smooth
right $G$ action such that:
\begin{enumerate}
  \item The action of $G$ preserves the fibers of $\pi$, and gives each fiber
  $P_x \defeq \pi\inv(x)$ the structure of a \ib{right $G$-torsor}, i.e. the
  action of $G$ on $P_x$ is free and transitive.
  \item For every point $x \in X$, there exists a \ib{local trivialization} of
  $P$, i.e. a diffeomorphism $\varphi : P\vert_U \defeq \pi\inv(U) \to U \times G$
  that is $G$-equivariant (where the action on $U \times G$ is right multiplication
  with the second factor) and the following diagram commutes:
  \[\begin{tikzcd}
  P\vert_U \ar[rr, "\varphi"] \ar[dr, "\pi"'] && U \times G \ar[dl] \\
  & U
  \end{tikzcd}\]
  where the map $U\times G \to U$ is projection onto the first factor.
\end{enumerate}
\end{defn}
%
\begin{exmp}
Let $E \to X$ be a real vector bundle of rank $k$. For $x \in X$, let $\B_x$
denote the set of all bases of the fiber $E_x$, i.e. the set of linear isomorphisms
$\R^k \to E_x$. This has a natural right action of $\GL_k\R$ by precomposition.
Furthermore, this action is free and transitive, giving $\B_x$ the structure
of a $\GL_k\R$-torsor. Then let
\[
\B_{\GL_k\R}(E) \defeq \coprod_{x \in X}\B_x
\]
Using local trivializations of the vector bundle $E$, we equip $\B_{\GL_k\R}(E)$
with the structure of a smooth manifold such that the map
$\pi : \B_{\GL_k\R}(E) \to X$ taking $\B_x$ to $x$ is a submersion. This gives
$\pi : \B_{GL_k\R} \to X$ the structure of a principal $\GL_k\R$-bundle, called
the \ib{frame bundle} of $E$,  where the local trivializations are defined in terms of
local trivializations of $E$.
\end{exmp}
%
\begin{exmp}
Let $E \to X$ be a rank $k$ vector bundle equipped with a fiber metric, i.e. a
smoothly  varying inner product on the fibers $E_x$. Then the
\ib{orthonormal frame bundle} of $E$, denoted $\B_{\mathrm{O}}(E)$, is the principal
$\mathrm{O}_k$-bundle where the fiber over $x \in X$ is the $\mathrm{O}_k$-torsor
of orthonormal bases for $E_x$.
\end{exmp}
%
A near identical story holds for complex vector bundles -- from any complex
vector bundle we get a principal $\mathrm{GL}_k\C$-bundle of frames, and if
we fix a Hermitian fiber metric, we get a principal $\mathrm{U}_k$-bundle of
orthonormal frames. \\

Principal bundles can be thought bundles of symmetries of some other fiber bundle,
which can be made precise using the notion of an associated bundle, which allows
one to construct fiber bundles out of principal bundles.
%
\begin{defn}
Let $P \to X$ be a principal $G$-bundle, and let $F$ be a smooth manifold with
right $G$ action. The \ib{associated fiber bundle}, denoted $P \times_G F$
(sometimes denoted $P \times^G F$) is the space
\[
P \times_G F \defeq (P \times F)/G
\]
where the right $G$-action on $F$ is the diagonal action, i.e.
$(p,f)\cdot g = (p\cdot g, f\cdot g)$.
\end{defn}
%
If instead we have a left $G$-action on $F$, we can turn it into a right action
by defining $f \cdot g \defeq g\inv \cdot f$. As the name suggests, $P \times_G F$ is
a fiber bundle.
%
\begin{exer}
Let $\pi : P \to X$ a principal bundle. Given a smooth right action of $G$ on
$F$, use local trivializations of $P \to X$ to show that the map taking an
equivalence class $[p,f]$ to $\pi(p)$ gives $P\times_G F$ the structure of a fiber
bundle over $X$ with model fiber $F$.
\end{exer}
%
In the case that the model fiber is a vector space $V$, and the action is
linear, the associated bundle $P \times_G V$ is a vector bundle.
%
\begin{exer} \enumbreak
\begin{enumerate}
  \item Let $E \to X$ be a rank $k$ vector bundle, and $\B_{\GL_k\R}(E)$ be its
  $\GL_k\R$-bundle of frames. Let $\rho : \GL_k\R \to \GL_k\R$ be the
  defining representation (i.e. the identity map). Show that the associated bundle
  $\B_{\GL_k\R}(E) \times_{\GL_k\R} \R^k$ is isomorphic to $E$.
  \item Further suppose that $E$ comes equipped with a fiber metric, and let
  $\B_{\mathrm{O}}(E)$  be its orthonormal frame bundle. The associated
  bundle $\B_{\mathrm{O}}(E) \times_{\mathrm{O}_k} \R^k$ is isomorphic to $E$
  by a near identical proof as the previous part. How can one recover the fiber
  metric?
  \item Let $\rho^* : \GL_k\R \to \GL_k\R$ denote the dual representation of the
  defining representation $\rho$, so $\rho(A) = (A\inv)^T$. Show that the associated
  bundle is isomorphic to the dual bundle $E^*$. In particular, this should
  illuminate the distinction between the tangent and cotangent bundles.
\end{enumerate}
\end{exer}
%
\begin{exmp}
There are two important examples of associated bundles that we'll need to discuss
the Yang-Mills equations.
\begin{enumerate}
  \item The bundle $\Ad P \defeq P \times_G G$ where $G$ acts on $G$ by conjugation.
  \item The bundle $\ad P \defeq P \times_G \g$ (also denoted $\g_P$) where
  the action is the adjoint action.
\end{enumerate}
Confusingly, the former is sometimes called the ``Adjoint bundle" and the
latter is sometimes called the ``adjoint bundle," which makes it admittedly
hard to distuinguish between them when speaking.
\end{exmp}
%
Associated bundles have another nice feature -- their sections have
a nice interpretation in terms of $G$-equivariant maps.
%
\begin{prop}
Let $E = P \times_G F$ be an associated fiber bundle, and let
$\Gamma(X,E)$ denote the space of global sections, i.e. the space of
smooth maps $f : X \to E$ such that $\pi \circ f = \id_X$, where
$\pi : E \to X$ denotes the projection map. Then there is a bijective correspondence
\[
\Gamma(X,E) \longleftrightarrow \set{G-\text{equivariant maps } P \to F}
\]
\end{prop}
%
\begin{proof}
Let $\sigma : X \to E$ be a section. Then define the map
$\widetilde{\sigma} : P \to F$ as follows: for $x \in X$, let $(p,f)$ be a
representative for $\sigma(x)$. Then define $\widetilde{\sigma}(x) \defeq f$. \\

In the other direction, let $\widetilde{\varphi} : P \to F$ be an equivariant
map. Then define the section $\varphi : X \to E$ by
$\varphi(x) = [p,\widetilde{\varphi}(p)]$ for any choice of $p \in P_x$.
\end{proof}
%
\begin{exer}
Verify that the map $\varphi$ defined above is well-defined. Verify the two
constructions above are inverses to each other.
\end{exer}
%
The main takeaway from the proposition is the motto that ``$G$-equivariant
objects on $P$ descend to objects on $X$." \\

Before we discuss connections on principal bundles, we introduce the concept of
vector bundled valued forms.
%
\begin{defn}
Let $E \to X$ be a vector bundle. An \ib{$E$-valued differential $k$-form} is
a section of $\Lambda^kT^*X\otimes E$. We denote the space of $E$-valued $k$-forms
by $\Omega^k_X(E)$. If $V$ is a fixed vector space, a \ib{$V$-valued differential
$k$-form} is a $X\times V$-valued $k$-form, and we let $\Omega^k_X(V)$ denote the
space of $V$-valued $k$-forms.
\end{defn}
%
In a local frame $\set{e_i}$ for $E$, an $E$-valued $k$-form $\omega$ can be
written uniquely as
\[
\omega = \omega^i \otimes e_i
\]
for $k$-forms $\omega^i \in \Omega^k_X$,so an $E$-valued $k$-form can be thought of
as a vector of $k$-forms. We will usually omite the tensor symbol, and simply
write $\omega = \omega^ie_i$. However, this does not transform tensorially with
respect to coordinate changes on $X$ unless $E$ is a trivial bundle. The
components of the vector transform tensorially with respect to coordinate changes,
but the vector itself changes according to the transition functions of the vector
bundle $E$. Given $E$-valued forms $\omega \in \Omega^k_X(E)$ and
$\eta \in \Omega^\ell_X(E)$, we define their wedge product in a local trivialization to
be
\[
\omega \wedge \eta \defeq  (\omega^i \wedge \eta^j)e_i \otimes e_j\
\]
which is an element of $\Omega^{k+\ell}_X(E \otimes E)$. \\

For the most part, we will be concerned with Lie algebra valued forms,
which are just $\g$-valued forms for a fixed Lie algebra $\g$. These
forms have some additional operations coming from the Lie algebra structure
of $\g$. Fix a basis $\set{\xi_i}$ for $\g$. This determines a global
trivialization of the trivial bundle $X \times \g$, so any $\g$-valued
$K$-form $\omega \in \Omega^k_X(\g)$ can be uniquely written as
$\omega = \omega^i\xi_i$. Let
$\omega \in \Omega^k_X(\g)$ and $\eta \in \Omega^\ell_X(\g)$. Then define their
bracket to be
\[
[\omega\wedge\eta] \defeq (\omega^i \wedge \eta^j)[\xi_i,\xi_j]
\]
In other words, it is the composition
\[\begin{tikzcd}
\Omega^k_X(\g) \otimes \Omega^k_X(\g) \ar[r] & \Omega^{k+\ell}_X(\g \otimes \g) \ar[r] &
\Omega^{k+\ell}(\g)
\end{tikzcd}\]
where the first map is the wedge product, and the second map is induced by the
Lie bracket. Finally, the usual exterior derivative $d : \Omega^k_X \to \Omega^{k+1}_X$
extends to an operator on $\Omega^k_X(\g)$, given by $d\omega = d\omega^i\xi_i$. \\

We now discuss connections. Let $\pi : P \to X$ be a principal $G$-bundle, and
let $\g$ be the Lie algebra of $G$. The projection map $\pi$ is a submersion,
so it is constant rank. Therefore, the subset $V \subset TP$ where the fiber
over $p$ is $\ker d\pi_p$ is a subbundle, called the \ib{vertical distribution}
of $P$, giving us an exact sequence of vector bundles over $P$
\[\begin{tikzcd}
0 \ar[r] & V \ar[r] & TP \ar[r] & \pi^*TX \ar[r] & 0
\end{tikzcd}\]
%
\begin{defn}
A \ib{connection} on $P$ is a distribution $H \subset TP$ such that
\begin{enumerate}
  \item $V \oplus H = TP$ \\
  \item $H_{p \cdot g} = d(R_g)_pH_p$, where $R_g : P \to P$ is the map
  $p \mapsto p\cdot g$.
\end{enumerate}
The distribution $H$ is also called the \ib{horizontal distribution}.
We let $\mathscr{A}(P)$ denote the space of connections on $P$.
\end{defn}
%
Equivalently, it is a choice of $G$-invariant splitting of the exact sequence.
The perspective of viewing a connection as a horizontal distribution is useful
at times, but it is often more convenient for computations to rephrase a connection
in terms of $\g$-valued forms. Let $\exp : \g \to G$ denote the exponential map.
Given $X \in \g$ and $p \in P$, the exponential map determines a curve
$\gamma_X$ with $\gamma_X(0) = p$ where
\[
\gamma_X(t) \defeq p \cdot \mathrm{exp}(tX)
\]
Since the action of $G$ preserves the fiber $P_{\pi(p)}$, the tangent vector
\[
\dt{\gamma}_X \defeq \frac{d}{dt}\bigg\vert_{t=0} \gamma_X(t)
\]
lies in $V_p$. Furthermore, since the action of $G$ on $P$ is free,
we have that $\dt{\gamma}_X = 0$ if and only if $X = 0$. Finally,
the mapping $X \mapsto \dt{\gamma}_X$ is linear, so we have that this
gives an isomorphism $\g \to V_p$ by a dimension count. Doing this
over all $p \in P$, this gives an isomorphism of $V$ with the trivial bundle
$P \times \g$. Because of this, we will implicitly identify elements of
$\g$ with the vertical vector fields they determine. One thing to note
is how these vector fields transform with respect to the action of $G$.
%
\begin{prop}
Let $X \in \g$, and let $\widetilde{X}$ denote the vertical vector field on $P$
induced by $X$. For $g \in G$, let $R_g : P \to P$ be map given by the action of $g$.
Then
\[
(R_g)_*\widetilde{X} = \widetilde{\Ad_{g\inv}X}
\]
\end{prop}
%
\begin{proof}
We compute
\begin{align*}
((R_g)_*\widetilde{X}_p)
&= (R_g)_*\left(\frac{d}{dt}\bigg\vert_{t=0}p\cdot\exp(tX)\right) \\
&= \frac{d}{dt}\bigg\vert_{t=0}p\cdot(\exp(tX)g) \\
&= (p\cdot g)(g\inv\exp(tX)g) \\
&= (\widetilde{\Ad_{g\inv}X})_{p\cdot g}
\end{align*}
\end{proof}
%
Furthermore, the identification of the vertical distribution $V$ with the trivial
bundle $X \times \g$ gives us a nice characterization of $E$-valued forms,
when $E = P\times_G W$ is an associated bundle coming from a linear representation
$\rho : G \to \GL(W)$.
%
\begin{prop}
Let $P \to M$ be a principal bundle and $E$ the associated bundle coming from
a linear representation $\rho : G \to \GL(W)$. Then there is a bijective
correspondence
\[
\Omega^k_X(E) \leftrightarrow \set{\alpha \in \Omega^k_P(W) ~:~ R^*_g\alpha =
\rho(g\inv)\alpha~,~ \iota_X\alpha = 0 \forall X \in \g}
\]
where we identify $X \in \g$ with its vertical vector field and
$\iota_\xi$ denotes interior multiplication.
\end{prop}
%
Morally, the correspondence comes from the fact that a form on $P$ descending
to $X$ should satisfy $G$-invariance, and should be invariant in the vertical
directions.
%
\begin{exer}
Prove the previous proposition.
\end{exer}
%
\begin{exer}
Using the proposition above, prove that the space $\mathscr{A}(P)$ of connections
is an affine space over $\Omega^1_X(\g_P)$, i.e. show that the difference
$A_1 - A_2$ between two connections is an element of $\Omega^1_X(\g_P)$.
\end{exer}
%
Now suppose we have a horizontal distribution $H \subset TP$. The decomposition
$TP = V \oplus H$ gives us a projection map $TP \to V$ with kernel $H$.
Identifying $V$ with $P \times \g$, the projection map can be idenfied with
a $\g$-valued $1$-form $A \in \Omega^1_P(\g)$, called the
\ib{connection $1$-form}. Using the transformation law for the vertical
vector fields determined by $\g$, we get the following transformation law
for the connection $1$-form $A$.
%
\begin{prop}
A connection $1$-form $A \in \Omega^1_P(\g)$ satisfies
\[
R^*_gA = \Ad_{g\inv}A
\]
\end{prop}
%
\begin{proof}
For $v \in T_pP$, decompose $v = X + h$ with $X \in \g$ and $h \in H_p$.
We then compute
\begin{align*}
(R_g^*A)(v) &= (R^*_gA)(X + h) \\
&= A_{p\cdot g}((R_g)_*X + (R_g)_*h) \\
&= A_{p\cdot g}(\Ad_{g\inv} X) \\
&= (\Ad_{g\inv}A)_p(X+h)
\end{align*}
where we use the fact that $H_p$ is the the kernel of $A_p$ and the fact that $H$
is $G$-invariant.
\end{proof}
%
Furthermore, since $A$ is given by projection onto the vertical distribution,
we have that $\iota_XA = X$ for all $X \in \g$. This gives us an identification
of $\mathscr{A}(P)$ with the subset of $\Omega^1_P(\g)$ satisfying the conditions
\begin{enumerate}
  \item $R_g^*A = \Ad_{g\inv}A$
  \item $\iota_XA = X$ for all $X \in \g$.
\end{enumerate}
%
The second condition can be rephrased in terms of the \ib{Maurer-Cartan form}
$\theta \in \Omega^1_G(\g)$, which is defined by
\[
\theta_g(v) = (dL_{g\inv})_g(v)
\]
where $L_{g\inv} : G \to G$ is left multiplication by $g\inv$. The action
of $G$ on any $G$-torsor $X$ gives us a Maurer-Cartan form on $X$.
%
\begin{exer}
Show that the second condition is equivalent to $A\vert_{P_x} = \theta$
for any $x \in X$.
\end{exer}
%
The Maurer-Cartan form $\theta$ satisfies the \ib{Maurer-Cartan equation}
\[
d\theta + \frac{1}{2}[\theta\wedge\theta] = 0
\]
%
\begin{defn}
Let $A \in \mathscr{A}(P)$ be a connection. Then the \ib{curvature of $A$},
denoted $F_A$, is the $\g$-valued $2$ form
\[
F_A \defeq dA = \frac{1}{2}[A \wedge A]
\]
\end{defn}
%
\begin{prop} \enumbreak
\begin{enumerate}
  \item $R^*_gF_A = \Ad_{g\inv}F_A$.
  \item $\iota_XF_A = 0$ for all $X \in \g$.
\end{enumerate}
\end{prop}
%
\begin{proof} \enumbreak
\begin{enumerate}
  \item We compute
  \begin{align*}
  R^*F_A &= R^*_gdA + \frac{1}{2}R^*_g[A \wedge A] \\
  &= d(\Ad_{g\inv}A) + \frac{1}{2}[R^*_gA \wedge R^*_GA] \\
  &= \Ad_{g\inv}dA + \frac{1}{2}[\Ad_{g\inv}A \wedge \Ad_{g\inv}A] \\
  &= \Ad_{g\inv}F_A
  \end{align*}
  \item For this, we use a lemma.
  \begin{lem*}[\ib{Cartan's magic formula}]
  Let $X$ be a vector field, and $\omega$ a $k$-form. Let $\mathcal{L}_X$ denote
  the Lie derivative along $X$. Then
  \[
  \mathcal{L}_X = d\iota_X\omega + \iota_Xd\omega
  \]
  \end{lem*}
  Let $X \in \g$, interpreted as a vertical vector field on $P$. Then
  we compute
  \[
  \iota_XF_A = \iota_XdA + \frac{1}{2}\iota_X[A \wedge A]
  \]
  We compute the two terms separately. Cartan's magic formula gives us that
  \[
  \mathcal{L}_XA = d\iota_XA + \iota_XdA
  \]
  Since $\iota_XF_A$ is the constant function with value $X$,
  we have that $d\iota_XF_A$ is $0$, so we get $\mathcal{L}_XF_A = \iota_XdF_A$.
  Using the definition of the Lie derivative, we compute
  \begin{align*}
  \mathcal{L}_XA &= \frac{d}{dt}\bigg\vert_{t=0}R^*_{\exp{tX}}A \\
  &= \frac{d}{dt}\bigg\vert_{t=0}\Ad_{\exp(tX)\inv}A  \\
  &= [-X, A]
  \end{align*}
  For the other term, we compute
  \[
  \frac{1}{2}\iota_X[A \wedge A] = \frac{1}{2}[\iota_XA \wedge A] = [X,A]
  \]
  adding these together gives us the desired result.
\end{enumerate}
\end{proof}
%
In other words, the curvature $F_A$ descends to a $\g_P$-valued $2$-form
on the base manifold $X$.
%
\begin{exer}
Given a connection $A$ on a principal bundle $P$, prove that the curvature
$F_A$ vanishes if and only if horizontal distribution $H$ defined by the
kernel of $A$ is integrable. To prove this, reformulate Frobenius' theorem
(a distribution is integrable if and only if it is involutive) in terms
of the vanishing of a tensor, and show that this tensor (up to sign) is equal
to $F_A$.
\end{exer}
%
For vector bundles, a more familiar definition of a connection involves
a first order operator on sections satisfying a Leibniz rule. Using
the language of principal bundles and associated bundles, we recover
this notion with the exterior covariant derivative.
%
\begin{defn}
Let $E = P\times_G W$ be the associated vector bundle obtained from a
linear representation $\rho : G \to \GL(W)$, and let $\dt{\rho} : \g \to \End(W)$
denote the associated Lie algebra representation. The \ib{exterior covariant derviative}
is the map
\begin{align*}
d_A : \Omega^p_X(E) &\to \Omega^{p+1}_X(E) \\
\psi &\mapsto d\psi + \dt\rho(A) \wedge \psi
\end{align*}
\end{defn}
%
\begin{exer}
Recall that a connection on a vector bundle $E \to X$ is given in a local
trivialization by $d + A$ for some $\End(E)$-valued $1$-form $A$.
Show that when $P = \B_{\GL_k\R}(E)$ is the frame bundle of a vector
bundle $E$, the exterior covariant derivative on sections agrees with
this definition.
\end{exer}
%
For the most part, we will be concerned with situation when the vector
bundle is $\g_P$, in which case, the formula is given by
\[
d_A\psi = d\psi + [A,\psi]
\]
Since $\mathscr{A}(P)$ is an affine space over $\Omega^1_X(\g_P)$, given
a connection $A \in \mathscr{A}(P)$ and a $\g_P$-valued $1$-form
$\eta \in \Omega^1_X(\g_P)$, we have that $A + \eta$ is also a connection.
It can be shown that the curvature of $A+\eta$ is given by
\[
F_{A + \eta} = F_A + \frac{1}{2}[\eta\wedge\eta] + d_A\eta
\]
In particular, if we take a line of connections $A + t\eta$ with $t \in \R$,
we have
\[
\frac{d}{dt}\bigg\vert_{t=0}F_{A+t\eta}
= \frac{d}{dt}\bigg\vert_{t=0}F_A + \frac{t^2}{2}[\eta\wedge\eta] + td_A\eta
= d_A\eta
\]
So the exterior covariant derivative on $\g_P$ measures the infitesimal
change of the curvature of $A$ in the direction $\eta$.
%
%
%
\section{The Yang-Mills Equations}
%
To discuss the Yang-Mills equations, we will restrict to compact Lie groups
$G$. As before, $X$ will denote an $n$-dimensional closed smooth manifold. \\

Since $G$ is compact, its Lie algebra $\g$ is semisimple, so the Killing
form $\langle\cdot,\cdot\rangle : \g \otimes \g \to \R$ is nondegenerate.
For the rest of our discussion, $\langle\cdot,\cdot\rangle$ can be replaced by
any inner product invariant under the Adjoint action, though it does us no
harm to assume that it is the Killing form.
%
\begin{lem}
Let $\langle\cdot,\cdot\rangle$ denote any Adjoint invariant inner product on $\g$.
Then for $X_1,X_2,X_3 \in \g$, we have
\[
\langle[X_1,X_2],X_3\rangle = \langle X_1,[X_2,X_3]\rangle
\]
\end{lem}
%
\begin{proof}
We compute
\begin{align*}
\langle[X_1,X_2],X_3\rangle &= \langle[-X_2,X_1],X_3\rangle \\
&= \frac{d}{dt}\bigg\vert_{t=0}\langle\Ad_{\exp(-tX_2)}X_1,X_3\rangle \\
&= \frac{d}{dt}\bigg\vert_{t=0}
\langle\Ad_{\exp{tX_2}}\Ad_{\exp(-tX_2)}X_1,\Ad_{\exp(tX_2)}X_3\rangle \\
&= \langle X_1,[X_2,X_3]\rangle
\end{align*}
\end{proof}
%
The form $\langle\cdot,\cdot\rangle$ induces a fiber metric on
$P \times \g$, and invariance under the Adjoint action tells us that
this fiber metric descends to a fiber metric on $\g_P$. This gives
us pairings
\begin{align*}
\Omega^k_X(\g_P) \otimes \Omega^\ell_X(\g_P) &\to \Omega^{k+\ell}_X \\
\omega \otimes \eta &\mapsto \langle\omega,\eta\rangle
\end{align*}
We now fix an orientation and a Riemannian metric $g$ on $X$. This gives us:
\begin{enumerate}
  \item A Riemannian volume form $\mathrm{Vol}_g \in \Omega^n_X$.
  \item A Hodge star operator $\star : \Omega^k_X \to \Omega^{n-k}_X$.
  \item Fiber metrics $\langle\cdot,\cdot\rangle_g$ on the bundles
  $\Lambda^kT^*X$.
\end{enumerate}
%
The Hodge star extends to $\g_P$-valued forms, which gives us inner products
on $\Omega^k_X(\g_P)$ given by
\[
(\omega,\eta) \defeq \int_X \langle\omega,\star\eta\rangle
\]
We let $\norm{\cdot}$ denote the norm induced by these inner products. \\

We now introduce the gauge group of a prinicipal $G$-bundle $P \to X$.
%
\begin{defn}
Let $\pi : P \to X$ be a principal $G$-bundle. The \ib{gauge group}, denoted
$\mathscr{G}(P)$, is the group of automorphisms of $P$, i.e. $G$-equivariant
diffeomorphisms $\varphi : P \to P$ such that $\pi = \pi \circ \varphi$.
An element of $\mathscr{G}(P)$ is called a \ib{gauge transformation}.
\end{defn}
%
\begin{prop}
The group $\mathscr{G}(P)$ is isomorphic to the group of sections
$\Gamma(X,\Ad P)$, where the group operation is pointwise multiplication.
\end{prop}
%
\begin{proof}
We provide maps in both directions. Suppose we have an automorphism
$\varphi : P \to P$. Since $\pi = \pi \circ \varphi$, the map $\varphi$ preserves
the fibers of $\pi$. Therefore, for any $p \in P$, we have that
$p$ and $\varphi(p)$ differ by the action of some $g_p \in G$. The mapping
$g_\varphi : P \to G$ taking $p \mapsto g_p$ is easilty verified to be
equivariant with respect to the conjugation action of $G$, so it defines
a section of $\Ad P$ \\

In the other direction, given a $G$-equivariant map
$f : P \to G$, we get a bundle automorphism $\varphi_f : P \to P$
where  $\varphi_f(p) = p \cdot f(p)$. The two maps we constructed are clearly inverse
to each other, giving us the desired correspondence.
\end{proof}
%
The gauge group $\mathscr{G}(P)$ acts on the space $\Omega^1_P(\g)$ of
$\g$-valued forms by pullback. We claim that it preserves the
subspace $\mathscr{A}(P) \subset \Omega^1_P(\g)$.
%
\begin{prop}
For a connection $A \in \mathscr{A}(P)$ and a gauge transformation
$\varphi : P \to P$, we have
\begin{enumerate}
  \item $R_g^*A = \Ad_{g\inv}\varphi^*A$.
  \item $\iota_X\varphi^*A = X$ for all $X \in \g$.
\end{enumerate}
Equivalently, if we let $g_\varphi : P \to G$ denote the equivariant map
associated to $\varphi$, we have
\[
\varphi^*A = \Ad_{g\inv_\varphi A} + g_\varphi^*\theta
\]
where $\theta \in \Omega^1_G(\g)$ is the Maurer-Cartan form.
\end{prop}
%
\begin{exer}
Prove the previous proposition.
\end{exer}
%
\begin{prop}
Let $A \in \mathscr{A}(P)$ be a connection, $\varphi : P \to P$ a gauge
transformation, and $g_\varphi : P \to G$ the associated equivariant map.
Then
\[
F_{\varphi^*A} = \Ad_{g_\varphi\inv}F_A
\]
\end{prop}
%
\begin{proof}
We compute
\begin{align*}
&F_{\varphi^*A} = d(\Ad_{g_\varphi\inv} A + g_\varphi^*\theta)
+ \frac{1}{2}[\Ad_{g_\varphi\inv} A + g_\varphi^*\theta\wedge
\Ad_{g_\varphi\inv} A + g_\varphi^*\theta] \\
&= \Ad_{g_\varphi\inv}dA + g_\varphi^*d\theta + \frac{1}{2}
\left([\Ad_{g_\varphi\inv}A\wedge \Ad_{g_\varphi\inv} A] + [\Ad_{g_\varphi\inv} A\wedge
g_\varphi^*\theta] + [g^*_\varphi\theta\wedge \Ad_{g_\varphi\inv}A]
+ [g_\varphi^*\theta\wedge g_\varphi^*\theta] \right) \\
&= \Ad_{g_\varphi\inv} dA + \frac{1}{2}[\Ad_{g_\varphi\inv}A, \Ad_{g_\varphi\inv} A]] \\
&= \Ad_{g_\varphi\inv F_A}
\end{align*}
%
The term
\[
g_\varphi^*d\theta + \frac{1}{2}[g_\varphi^*\theta\wedge g_\varphi^*\theta]
\]
vanishes due to the Maurer-Cartan equation. The term
\[
\frac{1}{2}\left([\Ad_{g_\varphi\inv} A\wedge
g_\varphi^*\theta] + [g^*_\varphi\theta\wedge \Ad_{g_\varphi\inv}A]\right)
\]
vanishes due to the fact that $[\cdot,\cdot]$ is skew symmetric on $1$-forms.
\end{proof}
%
With some of the preliminary results established, we arrive at the Yang-Mills
functional.
%
\begin{defn}
The \ib{Yang-Mills functional} is the map $L : \mathscr{A}(P) \to \R$ given by
\[
L(A) \defeq \norm{F_A}^2 = \int_X \langle F_A\wedge\star F_A\rangle
\]
\end{defn}
%
We note that for any gauge transformation $\varphi \in \mathscr{G}(P)$,
we have $L(\varphi^*A) = L(A)$, since we have
\[
L(\varphi^*A)
= \int_X\langle \Ad_{g_\varphi\inv}F_A\wedge\star\Ad_{g_\varphi\inv}F_A\rangle
= \int_X\langle F_A\wedge\star F_A\rangle = L(A)
\]
because of this we say that $L$ is \ib{gauge invariant}. \\

The Yang-Mills equations are the variational equations for the Yang-Mills
functional.
%
\begin{prop}[\ib{The first variation}]
Let $A$ be a local extremum of $L$. Then we have
\[
d_A\star F_A = 0
\]
\end{prop}
%
\begin{proof}

\end{proof}
%
%
%
\section{Holomorphic Vector Bundles and Yang-Mills Connections}
%
A good references for this material would be \cite{Kob} and \cite{McDuff}\\

Let $X$ denote a complex manifold.
%
\begin{defn}
A \ib{holomorphic vector bundle} is a complex vector bundle $\pi : E \to X$
such that the total space $E$ is a complex manifold and $\pi$ is holomorphic.
\end{defn}
%
Given a holomorphic vector bundle $E \to X$, we can find a trivialization
of $E$ such that the transition functions are holomorphic. In a neighborhood
$U \subset X$ such that $E\vert_U$ is trivial, the smooth sections can
be identified with functions $U \to \C^n$, and the holomorphic sections
can be identified with the holomorphic functions $U \to \C^n$. We
have a local operator $\dbar$, which we can apply componentwise to a local
section to get an operator on smooth sections over $U$. Furthermore, since
$\dbar$ annihilates holomorphic functions and the transition functions
are holomorphic, we have that $\dbar$ glues to a well defined operator
$\dbar_E : \mathcal{A}^0_X(E) \to \mathcal{A}^{0,1}_X(E)$. The holomorphic sections of $E$
are then exactly the sections annihilated by $\dbar_E$. Furthermore, the operator
$\dbar_E$ extends to operators $\dbar_E : \mathcal{A}^k_X(E) \to \mathcal{A}^{k+1}_X(E)$,
and satifies the condition $\dbar_E^2 = 0$, since $\dbar^2 = 0$. The punchline
is that the holomorphic structure on $E$ is entirely determined by this operator.
%
\begin{thm}
Let $\pi : E \to X$ be a $C^\infty$ complex vector bundle, and let
$D : \mathcal{A}^0_X(E) \to \mathcal{A}^{0,1}(E)$ be an operator satifying
$D^2 = 0$. Then there exists a unique complex structure on $E$
such that $\pi$ is holomorphic and $D$ coincides with the operator $\dbar_E$.
\end{thm}
%
This can be seen as a linearized version of the Newlander-Nirenberg theorem.
In particular, a holomorphic vector bundle $E \to X$ can be thought
of as a smooth vector bundle along with a choice of operator $\dbar_E$.
Since the operator $\dbar$ satisfies a Leibniz rule, the operator $\dbar_E$
behaves like a connection. In a \emph{smooth} local trivialization,
we can write
\[
\dbar_E = \dbar + B
\]
where $B$ is a smooth $\mathrm{M}_n\C$-valued $(0,1)$-form. Indeed,
we have that the space of holomorphic structures on a smooth vector bundle
$E \to X$ is an affine space over $\mathcal{A}^{1,0}(\End E)$. We
let $\mathscr{C}(E)$ denote the space of holomorphic structures on $E$.\\

We now restrict to the case where $X$ is a Riemann surface.
%
\begin{defn}
The \ib{slope} of a holomorphic vector bundle $E \to M$ is
\[
\mu(E) \defeq \frac{c_1(E)}{\mathrm{rank}(E)}
\]
where we think of $c_1(E) \in H^2(X,\Z)$ as an integer via integration over $X$.
\end{defn}
%
Sometimes the integer $c_1(E)$ is also referred to as the \ib{degree} of $E$.
One thing to note is that the slope of a holomorphic vector bundle is
independent of the holomorphic structure -- both the degree and rank are
topological invariants, and only depend on the underlying $C^\infty$ complex
vector bundle.
%
\begin{defn}
A holomorphic vector bundle $E \to X$ is
\begin{enumerate}
  \item \ib{Stable} if for every holomorphic subbundle $F \subset E$,
  we have $\mu(F) < \mu(E)$.
  \item \ib{Semistable} if for every holomorphic subbundle $F \subset E$,
  we have $\mu(F) \leq \mu(E)$.
  \item \ib{Unstable} if $E$ is not semistable.
\end{enumerate}
\end{defn}
%
While the slope is a topological invariant, stability is not, since
we only consider holomorphic subbundles -- which depend on the holomorphic structure.
We also note that both the degree and rank are additive in exact sequences, which
immediately gives us:
%
\begin{prop}
Suppose we have the short exact sequence of holomorphic bundles
\[\begin{tikzcd}
0 \ar[r] & E \ar[r] & F \ar[r] & G \ar[r] & 0
\end{tikzcd}\]
Then we have
\[
\mu(F) = \frac{\mathrm{deg}(E) + \mathrm{deg}(G)}{\mathrm{rank}(E) + \mathrm{rank}(G)}
\]
\end{prop}
%
\begin{cor}
Given a short exact sequence of holomorphic bundles
\[\begin{tikzcd}
0 \ar[r] & E \ar[r] & F \ar[r] & G \ar[r] & 0
\end{tikzcd}\]
If $\mu(E) \geq \mu(F)$, then $\mu(F) \geq \mu(G)$. Likewise, if $\mu(E) \leq \mu(F)$,
then $\mu(F) \leq \mu(G)$.
\end{cor}
%
In other words, slopes behave monotonically in short exact sequences.
The terminology comes from Geometric Invariant Theory (GIT). The main
result will use is:
%
\begin{thm}[\ib{The Harder-Narasimhan Filtration}]
Let $E \to X$ be a holomorphic vector bundle. Then $E$ admits a canonical filtration
\[
0 = E_0 \subset E_1 \subset \cdots \subset E_n = E
\]
by holomorphic subbundles $E_i$ such that $E_i/E_{i-1}$ is semistable and
\[
\mu(E_1/E_0) > \mu(E_2/E_1) > \cdots > \mu(E_n/E_{n-1})
\]
\end{thm}
%
The proof of the above theorem is not extremely difficult, but we omit it.
The main idea is that any holomorphic vector bundle has a unique
maximal semistable subbundle, which we take to be $E_1$. We then take
$E_2$ to be the preimage of the maximal semistable bundle of $E_1/E_0$ under
the quotient map, and continue inductively. The slopes
$\mu_i \defeq \mu(E_i/E_{i-1})$ gives us $n$ rational numbers. If $k$ denotes
the rank of $E$, then we construct an element of $\Q^k$ by arranging
the $\mu_i$ in order, and repeating the entry $\mu_i$ a total of
$\mathrm{rank}(E_i/E_{i-1})$ times. We call this vector the
\ib{Harder-Narasimhan type} of $E$. \\

Our ultimate goal will be to relate moduli spaces of holomorphic vector
bundles over $X$ to Yang-Mills connections. To see this, let $E \to X$ be
a $C^\infty$ complex vector bundle of rank $n$, and fix a Hermitian metric
on $E$. Then let $P \to X$ denote the principal $\mathrm{U}_n$-bundle of frames
for $E$. We abbreviate the gauge group $\mathscr{G}(P)$ as $\mathscr{G}$.
%
\begin{prop}
There is a bijection $\mathscr{A}(P) \leftrightarrow \mathscr{C}(E)$.
\end{prop}
%
\begin{proof}
We provide maps in both directions. Suppose we have a connection
$A \in \mathscr{A}(P)$. Then $\mathcal{A}$ induces a covariant derivative
$d_A : \mathcal{A}^0_X(E) \to \mathcal{A}^1_X(E)$. The $(0,1)$ part of $d_A$
automatically satisfies $(d_A^{0,1})^2 = 0$, since $\mathcal{A}^2_X = 0$ by
dimension reasons. Therefore, $d_A^{0,1}$ defines a holomorphic structure
on $E$. \\

In the other direction, given a holomorphic structure $\dbar_E$,
there exists a unique Hermitian connection $A$ such that $d_A^{1,0} = \dbar_E$
called the \ib{Chern connection}, which is a sort of analogue to the Levi-Civita
connection in Riemannian geometry.
\end{proof}
%
Let $\mathscr{G}_\C$ denote the group of smooth bundle automorphisms of $E$.
Though both $\mathscr{G}_\C$ and $\mathscr{G}$ are both infinite dimensional,
the former can be seen as the complexification of the latter. The space
$\mathscr{C}(E)$ has a natural action by $\mathscr{G}_\C$ by conjugation.
Furthermore, the orbits under this action are exactly the isomorphism
classes of holomorphic structures on $E$. This is most easily seen by
characterizing an isomorphism $\varphi : E \to F$ of holomorphic vector bundles
as a smooth bundle isomorphism intertwining $\dbar_E$ and $\dbar_F$. However,
the na\"ive quotient $\mathscr{C}(E)/\mathscr{G}_\C$ is poorly behaved
(for example, it is not Hausdorff). To remedy this, we restrict our attention
to semistable bundles. \\

The relationship between $\mathscr{G}_\C$ and $\mathscr{G}$ as well
as the identification of $\mathscr{A}(P)$ and $\mathscr{C}(E)$ suggests
that isomorphism classes of holomorphic bundles should have something to
do with gauge equivalence classes of connections on $P$. This is turns out
to be true, and is an infinite dimensional version of the relationship between
a GIT quotient and a symplectic quotient. To investigate further, we make a short
digression regarding this relationship. \\

Let $G$ be a reductive complex group, and $X$ a K\"ahler manifold with K\"ahler
metric $\omega$, equipped with a ``nice" action of $G$. In the usual setting,
$X$ is a smooth projective variety with a fixed embedding $X \hookrightarrow \CP^N$,
the K\"ahler metric $\omega$ is the restriction of the Fubini-Study form,
and the $G$-action is induced by a homomorphism $G \to \GL_{N+1}(\C)$. In general,
the na\"ive quotient $X/G$ is not well behaved, and one restricts the action to a
subset $X_{ss}$ consisting of \ib{semistable points} to construct the
\ib{GIT quotient} $X_{ss}/G$. \\

Then let $K \subset G$ denote the maximal compact subgroup, which has the
property that its complexification is isomorphic to $G$. Suppose that
the action of $K$ on $X$ is symplectic, i.e. the action of any $k \in K$
preserves the K\"ahler metric on $X$. Let $\mathfrak{k}$ denote the
Lie algebra of $K$. Then the infinitesimal action of $K$ is given by
the Lie algebra homomorphism $\mathfrak{k} \to \mathfrak{X}(X)$ (where
$\mathfrak{X}(X)$ denotes the space of vector fields on $X$) defined by
$\xi \mapsto X_\xi$ where
\[
(X_\xi)_p \defeq \frac{d}{dt}\bigg\vert_{t=0}\mathrm{exp}(t\xi)\cdot p
\]
\begin{defn}
A symplectic action of $K$ on $X$ is \ib{Hamiltonian} if for each
$\xi \in \mathfrak{k}$, there exists a function $H_\xi : X \to \R$
such that for all $p \in X$ and $v \in T_pX$, we have
\[
\omega_p((X_\xi)_p,v) = (dH_\xi)_p(v)
\]
and the mapping $\xi \mapsto H_\xi$ is $K$-equivariant with respect
to the right actions of $K$ on $\mathfrak{k}$ by the Adjoint action
and precomposition with left translation $L_k$ on $C^\infty(X)$. The
functions $H_\xi$ are called \ib{Hamiltonian functions}.
\end{defn}
%
\begin{defn}
Suppose we have a Hamiltonian action of $K$ on $X$. A \ib{moment map}
for the action is a $K$-equivariant map $X \to \mathfrak{k}^*$ (where the
action on $\mathfrak{k}$ is the coadjoint action) such
that for any $p \in X$, $v \in T_pX$, and $\xi \in \mathfrak{k}$, we have
\[
d\mu_p(v)(\xi) = \omega_p((X_\xi)_p,v)
\]
\end{defn}
%
One things to note is that the Hamiltonian functions can be
recovered by the moment maps. If a Hamiltonian action admits a moment map,
then
\[
H_\xi(p) = \mu(p)(\xi)
\]
The let $\langle\cdot,\cdot\rangle$ be an inner product on $\mathfrak{k}^*$
that is invariant under the coadjoint action, and $\norm{\cdot}$ the induced
norm. Since $X$ is compact, the map $\norm{\mu}^2 : X \to \R$ attains
its minimum, and WLOG we assume that the minimum value is $0$.
%
\begin{defn}
The \ib{symplectic quotient} of $X$ by $K$ is the quotient space
\[
\mu\inv(0)/K
\]
\end{defn}
%
The symplectic quotient can also be referred to as the \ib{symplectic reduction}
or the \ib{Marsden-Weinstein quotient}.
%
\begin{thm}
The symplectic quotient of $X$ by $K$ admits a unique K\"ahler structure
such that the K\"ahler metric on $\mu\inv(0)/K$ is induced by the K\"ahler
metric on $X$.
\end{thm}
%
The relationship between the GIT quotient and the symplectic quotient is
given by the Kempf-Ness theorem.
%
\begin{thm}[\ib{Kempf-Ness}]
Suppose a complex reductive group $G$ acts on a K\"ahler manifold $X$
such that the action of the maximal compact subgroup $K \subset G$ is
Hamiltonian and admits a moment map $\mu : X \to \mathfrak{k}^*$. Then
the $G$-orbit of any semistable point contains a unique $K$-orbit minimizing
$\norm{\mu}^2$. This establishes a homeomorphism
\[
X_{ss}/G \longleftrightarrow \mu\inv(0)/K
\]
\end{thm}
%
We now want to relate the previous discussion to our situation. Using
the identification of $\mathscr{A}(P)$ and $\mathscr{C}(E)$, we
want the action of $\mathscr{G}_\C$ to play the role of the complex reductive
group $G$ and the gauge group $\mathscr{G}$ to play the role of the maximal
compact subgroup. Since the space $\mathscr{A}(P)$ is infinite dimensional,
along with the groups $\mathscr{G}_\C$ and $\mathscr{G}$, we are working
in an infinite dimensional setting, but we will gloss over the analytic
details and work with them formally. \\

Our first task is to realize $\mathscr{A}(P)$ as a ``K\"ahler manifold."
Since $X$ is a surface, the Hodge star maps $\mathcal{A}^1_X(\g_P)$ to
itself and squares to $-1$, so it defines a ``complex structure" on
$\mathscr{A}(P)$, where we use the fact that $\mathscr{A}(P)$ is affine
over the vector space $\mathcal{A}^1_X(\g_P)$ to identify the ``tangent space"
of $\mathscr{A}(P)$ at a connection $A$ with $\mathcal{A}^1_X(\g_P)$.
Furthermore, the fact that for $1$-forms $\omega,\eta \in \mathcal{A}^1_X(\g_P)$
the pairing $\langle\omega\wedge\eta\rangle$ is skew-symmetric, we
can identify the pairing
\[
\omega \otimes \eta \mapsto \int_X \langle\omega\wedge\eta\rangle
\]
as a ``symplectic form" on $\mathscr{A}(P)$. Together, these give
$\mathscr{A}(P)$ the structure of a ``K\"ahler manifold." \\

Our next task is to show that the action of $\mathscr{G}$ on $\mathscr{A}(P)$ is
``Hamiltonian" with respect to this K\"ahler structure. One can
identify the ``Lie algebra" of $\mathscr{G}$ with the space of
sections $\Gamma(X,\g_P)$.
%
\begin{prop}
The infinitesimal action of $\phi \in \Gamma(X,\g_P)$ on $\mathscr{A}(P)$ is
given by the mapping $A \mapsto d_A\phi$.
\end{prop}
%
\begin{proof}
We compute the vector field at a connection $A \in \mathscr{A}(P)$ to be
\begin{align*}
&\frac{d}{dt}\bigg\vert_{t=0} \Ad_{\exp(t\phi)\inv} A + \exp(t\phi)^*\theta
= -[\phi, A] + \frac{d}{dt}\bigg\vert_{t=0}(dL_{\exp(-t\phi)} d(\exp(t\phi))) \\
&= [A,\phi] + \left(\frac{d}{dt}\bigg\vert_{t=0} dL_{\exp(-t\phi)}\right)d(\exp(0))
+ dL_{\exp(0)}\left(\frac{d}{dt}\bigg\vert_{t=0}d(\exp(t\phi))\right) \\[5pt]
&= [A,\phi] + d\phi\\
&= d_A\phi
\end{align*}
where for the third equality we use the product rule, and in the fourth equality
we use the fact that $\exp(0) = \id$ and that the derivative of
$\exp(t\phi)$ as $t \to 0$ is $\phi$.
\end{proof}
%
\begin{prop}
Let $\phi \in \Gamma(X,\g_P)$. Then the function
\begin{align*}
H_\phi : \mathscr{A}(P) &\to \R \\
A &\mapsto \int_X\langle F_A\wedge\phi\rangle
\end{align*}
is a Hamiltonian function for $\phi$.
\end{prop}
%
\begin{exer}
Prove the previous proposition.
\end{exer}
%
Since $\langle\cdot,\cdot\rangle$ is invariant under the adjoint action,
the mapping $\phi \mapsto H_\phi$ is clearly $\mathscr{G}$ equivariant,
so this tells us that the action is Hamiltonian. Furthermore,
the computation we made identifies the mapping $A \mapsto F_A$ as the
moment map for this action. To summarize, we have the following
analogies
%
\begin{align*}
\text{K\"ahler manifold }X &\longleftrightarrow \mathscr{A}(P) \\
\text{Complex reductive group }G &\longleftrightarrow \mathscr{G}_\C \\
\text{Maximal compact subgroup }K \subset G &\longleftrightarrow \mathscr{G} \\
\text{Moment map }\mu &\longleftrightarrow A \mapsto F_A \\
\text{Norm square of the moment map} \norm{\mu}^2 &\longleftrightarrow L
\end{align*}
%
The last missing piece is something analogous to the Kempf-Ness theorem.
%
\begin{thm}[\ib{Narasimhan-Seshadri}]
Let $\mathscr{A}_s(P) \subset \mathscr{A}(P)$ denote the subspace of connections
that are absolute minimal for the Yang-Mills functional, and correspond
to irreducible representations $\Gamma_\R \to \mathrm{U}_n$. Let $\mathscr{C}_s(E)$
denote the subspace of stable holomorphic structures on $E$. The isomorphism
classes of holomorphic bundles in $\mathscr{C}_s(E)$ admit unique
Yang-Mills connections (up to gauge equivalence) minimizing the Yang-Mills functional.
In other words, there is a homeomorphism
\[
\mathscr{A}_s(P)/\mathscr{G} \longleftrightarrow \mathscr{C}_s(E)/\mathscr{G}_\C
\]
\end{thm}
%
\begin{rem*}
The original proof is more algebraic in flavor. A proof more in the spirit
of the Atiyah-Bott paper was given by Donaldson in \cite{donaldson1983}.
The spirit of this proof is carried on by the proof of Hermitian-Yang-Mills
and the nonabelian Hodge theorem, which were both grew out of the
developments from the Atiyah-Bott paper.
\end{rem*}
%
One issue is that the Narasimhan-Seshadri theorem only works for
stable bundles. However, in the case that the rank and degree of $E$
are coprime, stability and semistability coincide for numerical reasons.
Our computations for the cohomology of the moduli space of holomorphic
bundles of rank $n$ and degree $k$ will include the assumption that the
rank and degree of $E$ are coprime. \\
%

In the finite dimensional case, when things are sufficiently, nice,
the function $f = \norm{\mu}^2$ is an equivariant Morse-Bott function,
which gives a stratification of the space $X$. Using Mayer-Vietoris, along
with some other algebraic topology and equivariant cohomology, one can
use this stratification to compute the equivariant cohomology of
the space $X$ with respect to the equivariant cohomology of the strata.
When the $G$ action is free, this tell us the regular cohomology of
the quotient space $X/G$. In our situation, the Yang-Mills functional
plays the role of the norm-squared of the moment map, and one would hope
that the analysis needed to do equivariant Morse theory isn't too
hard. Unfortunately, the analysis is \emph{very} hard. However, not all is
lost. If we can find a nice stratification of $\mathscr{C}(E) = \mathscr{A}(P)$
that \emph{looks like} it came from a nice equivariant Morse function, then
we can still do the cohomology computations. In our case, we have a
candidate for such a stratification. For a fixed Harder-Narasimhan type
$\mu$, let $\mathscr{C}_\mu(E)$ denote the subspace of holomorphic
structures on $E$ whose Harder-Narasimhan filtration has type $\mu$.
This subspace is preserved by the action of $\mathscr{G}_\C$, and
together all these subspaces give a stratification of the space
$\mathscr{C}(E)$ called the \ib{Harder-Narasimhan stratification}.
%
%
%
\section{Equivariant Cohomology}
%
References for this material include \cite{TuEC} and \cite{Kirwan}. \\

Our goal now is to compute the cohomology of the moduli space
of semistable holomorphic bundles over $X$ of rank $n$ and degree $k$,
denoted $N(n,k)$, in the case that $n$ and $k$ are coprime. Recall that
$N(n,k)$ has a global quotient description as
$N(n,k) = \mathscr{C}_{ss}(E)/\mathscr{G}_\C$. The computation of
the cohomology of $N(n,k)$ uses the concept of \ib{equivariant cohomology},
which is a cohomology theory for spaces with the action of a group. Before
doing so, we give a bit of exposition on where equivariant cohomology fits
into our story. \\

Recall that we mentioned that a $\mathrm{U}_n$-Yang-Mills connection is
equivalent to the choice of some Hermitian matrix, with some conditions
on the eigenvalues. Using the correspondence between unitary connections
and holomorphic structures, these conditions actually reflect the
Harder-Narasimhan filtration of the corresponding holomorphic bundle.
Using the perspective of unitary connections, Atiyah and Bott managed to show that
the Harder-Narasimhan stratification is \ib{equivariantly perfect},
i.e. looks like it came from an equivariantly perfect Morse function. In
other words, we can understand the equivariant cohomology of $\mathscr{C}(E)$
as being built from a simple formula involving the equivariant cohomology
of the strata $\mathscr{C}_\mu(E)$. This gives us a formula
for the equivariant cohomology of the semistable strata, from which
we can compute the ordinary cohomology of the quotient $N(n,k)$.
%
\begin{defn}
Let $G$ be a Lie group. A \ib{classifying space} for $G$ is a topological
space $BG$ such that we have a functorial correspondence
\[
\set{\text{Principal }G\text{-bundles } P \to X} \longleftrightarrow
\set{\text{Homotopy classes of maps }X \to BG}
\]
\end{defn}
%
Suppose $EG$ is a contractible space with a free action of $G$. Then the quotient
map $EG \to EG/G$ is a principal $G$-bundle.
%
\begin{thm}
The quotient space $BG = EG/G$ is a classifying space for $G$, which assigns to
a homotopy class of maps $[f] : X \to BG$ the principal $G$-bundle $f^*EG$,
which is independent of our choice of representative of $f$. Furthermore,
The spaces $EG$ and $EG/G$ are unique up to homotopy equivalence.
\end{thm}
%
Since the spaces $EG$ and $BG$ are only well-defined up to homotopy equivalence,
we will call a specific choice of $EG$ and $BG$ a \ib{model} for $EG \to BG$.
%
\begin{defn}
Let $G$ be a Lie group and $X$ a smooth manfold with an action of $G$. The
\ib{equivariant cohomology} of $X$, denoted $H^\bullet_G(X,\Z)$, is the
ordinary cohomology of the total space of the associated bundle
$EG\times_G X \to BG$, i.e.
\[
H^\bullet_G(X,\Z) \defeq H^\bullet(EG\times_G X,\Z)
\]
\end{defn}
%
We note that since any two models for $EG \to BG$ are homotopy equivalent,
this is well defined. In addition, we can define the equivariant cohomology over
any other coefficient group we want. Intuitively, the equivariant cohomology of
$X$ is is something like the regular cohomology of the quotient space $X/G$, but
this isn'texactly true when the action isn't nice. The difference is essentially
that the equivariant cohomology keeps track of stabilizer subgroups. To
see this, note that we have a map $EG\times_G X \to X/G$ taking an
equivalence class $[p,x] \in EG\times_G X$ to the orbit $[x] \in X/G$.
%
\begin{prop}
The fiber of $EG\times_G X \to X/G$ over $[x]$ is the quotient space $EG/G_x$, where
$G_x$ is the stabilizer of any representative of $[x]$.
\end{prop}
%
\begin{proof}
The fiber over $[x]$ is the subset $\set{[p,x\cdot g] ~:~ g \in G}\subset EG \times_G X$.
equivalently, this is the subset $\set{[p\cdot g\inv,x] ~:~ g \in G}$. Using
representatives of this form then gives the desired result.
\end{proof}
%
We note that since $EG$ is contractible and the action of $G_x \subset G$
is free on $EG$, the fiber over $[x]$ is a model for $BG_x$.
%
\begin{cor}
If the action of $G$ on $X$ is free, then the map $EG\times_G X \to X/G$
is a homotopy equivalence.
\end{cor}
%
\begin{cor}
If $X$ is contractible, then $H^\bullet_G(X,\Z) = H^\bullet(BG,\Z)$.
\end{cor}
%
\begin{proof}
The map $EG\times_G X \to X/G$ is a fiber bundle with contractible fiber $EG$.
\end{proof}
%
We now return to the computation of the cohomology of $N(n,k)$. The heavy
lifting for this computation comes from the following theorems:
%
\begin{thm}
Let $E \to X$ be a holomorphic vector bundle with Harder-Narasimhan filtration
\[
0 = E_0 \subset E_1 \subset \cdots \subset E_r = E
\]
Choose a $C^\infty$ splitting of the Harder-Narasimhan filtration, giving us a direct
sum decomposition as $C^\infty$ bundles
\[
E = \bigoplus_i D_i
\]
such that
\[
E_i = \bigoplus_{j < i} D_j
\]
Then we have
\[
H^\bullet_{\mathscr{G}_\C}(\mathscr{C}_\mu(E),\Q) \cong
\bigotimes_{i=1}^r H^\bullet_{\mathscr{G}_\C(D_i)}(\mathscr{C}_{ss}(D_i),\Q)
\]
where $\mathscr{G}_\C(D_i)$ denotes the group of smoot bundle automorphisms of $D_i$.
\end{thm}
%
\begin{thm}
Let $k_\mu$ denote the real codimension of $\mathscr{C}_\mu(E)$ inside
of $\mathscr{C}(E)$, and let $P_{t,\mathscr{G}_\C}(X)$ denote the
$\mathscr{G}_\C$-equivariant Poincar\'e series for a space $X$, i.e.
\[
P_{t,\mathscr{G}_\C}(X) \defeq \sum_i (\dim H^\bullet_{\mathscr{G}_\C}(X,\Q))t^i
\]
Then we have
\[
P_{t,\mathscr{G}_\C}(\mathscr{C}(E))
= \sum_\mu t^{k_\mu}P_{t,\mathscr{G}_\C}(\mathscr{C}_\mu(E))
\]
\end{thm}
%
\begin{thm}
The Poincar\'e seires for $B\mathscr{G}_\C$ is
\[
P_t(\mathscr{G}_\C) = \frac{\prod^n_{k=1}(1+t^{2k-1})^2g}
{(1-t^2n)\prod_{k=1}^{n-1}(1-t^{2k})^2}
\]
where $g$ is the genus of $X$.
\end{thm}
%
The first theorem tell us that we can understand the $\mathscr{G}_\C$-equivariant
cohomology of the strata by understanding the $\mathscr{G}_\C$-equivariant
cohomology of the semistable strata for lower dimensional bundles,
which will give us an inductive procedure for computing the Poincar\'e series.
The second theorem tells us that the equivariant cohomology of
the entire space is a simple expression in terms of the equivariant cohomology
of the strata. Since $\mathscr{C}(E)$ is contractible, the third
theorem tells us that if we can compute the equivariant cohomology of
all the strata except for the semistable locus, then we can compute the
equivariant cohomology of the semistable strata. \\

We first compute the codimension of the strata $\mathscr{C}_\mu(E)$. To do
this, we will use some facts regarding infinitesimal variations of holomorphic
structures.
%
\begin{prop}
Let $E \to X$ be a holomorphic vector bundle. The infinitesimal variations
of the holomorphic structure on $E$ are given by the Dolbeault cohomology group
$H^1_{\dbar}(X, \End(E))$.
\end{prop}
%
We are being purposefully vague when we say ``infinitesimal variation of holomorphic
structure." Our main use for the result is to identify the normal directions
of the strata $\mathscr{C}_\mu(E)$. The isomorphism class of a holomorphic
structure on a $C^\infty$ vector bundle $E$ is given by a $\mathscr{G}_\C$-orbit
in $\mathscr{C}(E)$, and the infinitesimal variation can be interpreted as the
normal directions to this orbit. This gives us a way to compute the
codimension of the strata $\mathscr{C}_\mu(E)$. From this perspective,
the normal directions to $\mathscr{C}_\mu(E)$ consist of infinitesimal variations
that change the type of the Harder-Narasimhan filtration. Explicitly,
we have a holomorphic subbundle $\End'(E) \to \End(E)$ consisting of
holomorphic endomorphisms of $E$ that preserve the Harder-Narasimhan filtration.
Then we can identify $H^1_{\dbar}(X,\End'(E))$ with the infinitesimal varations
consisting of the directions tangent to $\mathscr{C}_\mu(E)$. Furthermore, if
we let $\End"(E)$ denote the quotient bundle $\End(E)/\End'(E)$,
we can identify $\End"(E)$ with the holomorphic bundle endomorphism that do
not preserve the Harder-Narasimhan filtration, which tells us that the complex
codimension of $\mathscr{C}_\mu(E)$ in $\mathscr{C}(E)$ is the dimension of
$H^1_{\dbar}(X,\End"(E))$. To compute this, we use Riemann-Roch.
%
\begin{thm}[\ib{Riemann-Roch}]
Let $E \to X$ be a holomorphic vector bundle, where $X$ is genus $g$, and let
$h^i(E) = \dim H^i_{\dbar}(X,E)$. Then
\[
h^0(E) - h^1(E) = c_1(E) + (1-g)\mathrm{rank}(E)
\]
\end{thm}
%
Because of Riemann-Roch, it suffices to compute the dimension of
$H^0_{\dbar}(X,E)$ to compute $H^1_{\dbar}(X,E)$, and we want to apply
this to the holomorphic bundle $\End"(E)$.
%
\begin{prop}
\[
H^0_{\dbar}(X,\End"(E)) = 0
\]
\end{prop}
%
\begin{proof}
An element $g \in H^0_{\dbar}(X,\End"(E))$ is a global holomorphic endomorphism of
$E$ that does not fix the Harder-Narasimhan filtration. By assuption,
there exists some subbundle $E_i$ with $i > 0$ in the filtration such that
$g(E_i) \not\subset E_i$. By minimality of $i$, we have that
$g(E_{i-1}) \subset E_{i-1}$. Then let $k$ be the smallest integer such
that $g(E_i) \subset E_k$. Then the restriction of $g$ to $E_k$ factors
through the quotients to a nontrivial bundle homomorphism
$E_i/E_{i-1} \to E_k/E_{k-1}$. We note that both $E_i/E_{i-1}$ and $E_k/E_{k-1}$
are semistable and satisfy $\mu(E_i/E_{i-1}) > \mu(E_k/E_{k-1})$ by the properties
of the Harder-Narasimhan filtration. Let $K \subset E_i/E_{i-1}$ be
the smallest holomorphic subbundle containing the kernel, and
$A \subset E_k/E_{k-1}$ the smallest holomorphic subbundle containing the image,
giving us the short exact sequence of holomorphic bundles
\[\begin{tikzcd}
0 \ar[r] & K \ar[r] & E_i/E_{i-1} \ar[r] & A \ar[r] & 0
\end{tikzcd}\]
Semistability of $E_i/E_{i-1}$ implies that $\mu(A) \leq \mu(E_k/E_{k-1})$,
so $\mu(A) < \mu(E_i/E_{i-1})$. However, semistability of $E_i/E_{i-1}$ also
implies that $\mu(K) \leq \mu(E_i/E_{i-1})$, which would imply that
$\mu(E_i/E_{i-1}) \leq \mu(A)$, a contradiction.
\end{proof}
%
To use Riemann-Roch, we must identify the rank and degree of $\End"(E)$.
Since both of these quantities are topological invariants, we may work
in the $C^\infty$ category. We first compute the degree.
Since $\End(E) \cong E^*\otimes E$, we have that
$\mathrm{deg}(E) = 0$, where we use the fact that the degree of a bundle
is the same as the degree of its determinant line, and the formula for
the determinant line of a tensor product of bundles. Then since the
degree is additive in exact sequences, we get
\[
\mathrm{deg}(\End'(E)) + \mathrm{deg}(\End"(E)) = 0
\]
We then compute $\mathrm{deg}(\End'(E))$, which will tell us $\mathrm{deg}(\End"(E))$.
Fix a smooth splitting of the Harder-Narasimhan filtration, giving us a
$C^\infty$ decomposition
\[
E = \bigoplus_i D_i
\]
This gives us the identification as smooth bundles
\[
\End'(E) = \bigoplus_{i \geq j}\hom(D_i,D_j)
\]
Then if we let $\mu(D_i) = k_i/n)i$, we get
\[
\mathrm{deg}(\End'(E)) = \sum_{i\geq j} k_jn_i - k_in_j
\]
where we use additivity of degree with respect to direct sums and
the identification of $\hom(D_i,D_j)$ with $D_i^*\otimes D_j$. In the
case $i = j$, we get $\hom(D_i,D_j) = \End(D_i)$, which has degree $0$,
so we get
\[
\mathrm{deg}(\End'(E)) = \sum_{i > j} k_jn_i - k_in_j
\]
Negating this gives the degree of $\End"(E)$. \\

For the rank, this comes easily from the $C^\infty$ decomposition
\[
\End"(E) = \bigoplus_{i > j}\hom(D_i,D_j)
\]
giving us
\[
\mathrm{rank}(\End"(E)) = \sum_{i < j} n_in_j
\]
Putting everything together gives us
\[
\dim(H^1_{\dbar}(X,\End"(E))) = \sum_{i > j}n_ik_j-k_in)j + n_in_j(g-1)
\]
which by our earlier discussion, is the complex codimension of the strata
$\mathscr{C}_\mu(E)$.
%
%
%
\section{The Cohomology of the Moduli Spaces $N(n,k)$}
%
As before, we let $E \to X$ denote a $C^\infty$ complex vector bundle of
rank $n$ and degree $k$, where $n$ and $k$ are coprime. Recall that this
implies the notions of stability and semistability for a holomorphic structure
on $E$ coincide in this case.\\

The theorems in the previous section can be used to compute the
equivariant cohomology $H^\bullet_{\mathscr{G}_\C}(\mathscr{C}_\mu(E),\Z)$
using an inductive procedure involving the semistable strata for lower dimensional
holomorphic bundle, which in turn lets us compute the $\mathscr{G}_\C$-equivariant
cohomology of $\mathscr{C}_{ss}(E)$. However, this does not tell the cohomology
of the quotient space $N(n,k) \defeq \mathscr{C}_{ss}(E)/\mathscr{G}_\C$, since
the action of $\mathscr{G}_\C$ on $\mathscr{C}_{ss}(E)$ isn't free. To compute
the cohomology, we must pass to a quotient of $\mathscr{G}_\C$ that acts freely,
and then compute the equivariant cohomology with respect to that group.
%
\begin{prop}
Let $\dbar_E$ be a stable holomorphic structure on $E$. Then the stabilizer
subgroup of $\dbar_E$ is the central subgroup $C^\times \subset \mathscr{G}_\C$.
\end{prop}
%
\begin{proof}
Clearly $\C^\times$ is contained in the stabilizer of a holomorphic structure,
so it suffices to show that any automorphism $g \in \mathscr{G}_\C$ fixing
$\dbar_E$ is multiplication by an element of $\C^\times$. Since $g$
fixes $\dbar_E$, we get a direct sum decomposition of $E$ as a holomorphic bundle
\[
E = E_1 \oplus \cdots \oplus E_\ell
\]
where the $E_i$ are eigenbundles of $g$. We then claim that this decomposition
has only one term, which would verify our claim. Since $\dbar_E$ is stable,
$\mu(E_1) < \mu(E)$. Similarly, we have that
$\mu(E_2 \oplus\cdots\oplus E_\ell) < \mu(E)$.
Furthermore, we have $E_1 \cong E/(E_2\oplus\cdots\oplus E_k)$.
The exact sequence
\[\begin{tikzcd}
0 \ar[r] & E_2\oplus\cdots\oplus E_\ell \ar[r] & E \ar[r] & E_1
\end{tikzcd}\]
then implies that the slope of $E_1$ is larger than the slope of $E$, a contradiction.
\end{proof}
%
This gives us
\[
H^\bullet(N(n,k),\Z) = H^\bullet_{\overline{\mathscr{G}}_\C}(\mathscr{C}_{ss}(E),\Z)
\]
where $\overline{\mathscr{G}}_\C \defeq \mathscr{G}_\C/\C^\times$. To
compute the $\overline{\mathscr{G}}_\C$-equivariant cohomology, we use:
\begin{enumerate}
  \item $\mathscr{G}_\C$ deformation retracts onto $\mathscr{G}$.
  \item $\overline{\mathscr{G}}_\C$ deformations retracts onto
  $\overline{\mathscr{G}} \defeq \mathscr{G}/\mathrm{U}_1$
\end{enumerate}
The first point tells us that
\[
H^\bullet_{\mathscr{G}_\C}(\mathscr{C}_{ss}(E),\Z)
\cong H^\bullet_\mathscr{G}(\mathscr{C}_{ss}(E),\Z)
\]
The second point tells us that we may replace $\overline{\mathscr{G}}_\C$ with
$\overline{\mathscr{G}}$ to compute $H^\bullet(N(n,k),\Z)$. To do this, we
must first understand the cohomology of the classifying space
$B\overline{\mathscr{G}}$. We need the folliwing theorem:
%
\begin{thm}[\ib{Leray-Hirsch}]
Let $E \to X$ be a fiber bundle with model fiber $F$ such that
the inclusion $F \hookrightarrow E$ of a fiber induces a surjection
in rational cohomology. Then
\[
H^\bullet(E,\Q) \cong H^\bullet(X,\Q) \otimes H^\bullet(F,\Q)
\]
\end{thm}
%
To use this, we use a functoriality property of classifying spaces.
The exact sequence
\[\begin{tikzcd}
1 \ar[r] & \mathrm{U}_1 \ar[r] & \mathscr{G} \ar[r] & \overline{\mathscr{G}} \ar[r] & 1
\end{tikzcd}\]
induces a fibration
\[\begin{tikzcd}
B \mathrm{U}_1 \ar[r] & B\mathscr{G} \ar[d] \\
& B\overline{\mathscr{G}}
\end{tikzcd}\]
To apply Leray-Hirsch, we want to show that the pullback map induced
by $B \mathrm{U}_1 \to B\mathscr{G}$ induces a surjection
\[
H^\bullet(B\mathscr{G},\Q) \to H^\bullet(B \mathrm{U}_1, \Q)
\]
To do this, we provide a group homomorphism $\overline{\mathscr{G}} \to \mathrm{U}_1$,
such that the composition $\mathrm{U}_1 \hookrightarrow \mathscr{G} \to \mathrm{U}_1$
give maps of classifying spaces $B\mathrm{U}_1 \to B\mathscr{G} \to B\mathrm{U}_1$
inducing an isomorphism $H^\bullet(B \mathrm{U}_1,\Q) \to H^\bullet(B \mathrm{U}_1,\Q)$.
Fix a point $x \in X$, and let $g \in \mathscr{G}$, which we interpret as a
smooth bundle automorphism of $E$ preserving the Hermitian metric. Restricting $g$
to the fiber $E_x$ and taking the derterminant gives us our group homomorphism
$\mathscr{G} \to \mathrm{U}_1$. Since $E$ is rank $n$, this is a degree $n$
map. Then pullback induced map $B \mathrm{U}_1 \to B\mathscr{G} \to B \mathrm{U}_1$
multiplies the generator of $H^\bullet(B \mathrm{U}_1, \Q) \cong \Q[x]$ by $n$,
which is an isomorphism. Therefore, the map $B \mathrm{U}_1 \to B\mathscr{G}$
induces an isomorphism on rational cohomology. We then note that the
Poincar\'e series for $B \mathrm{U}_1$ is
\[
P_t(B \mathrm{U}_1) = \frac{1}{1-t^2} = 1 + t^2 + t^4 + \cdots
\]
So an application of Leray-Hirsch gives us
\[
P_t(B\overline{\mathscr{G}}) = P_t(\mathscr{G})(1-t^2)
\]
The next thing to do is to investigate the relationship between
$\mathscr{G}$-equivariant cohomology and $\overline{\mathscr{G}}$-equivariant
cohomology. Let $M$ be any $\overline{\mathscr{G}}$-space. The quotient map
$\mathscr{G} \to \overline{\mathscr{G}}$ gives $M$ the structure of
a $\mathscr{G}$-space. Furthermore, it induces a map
$B\mathscr{G} \to B\overline{\mathscr{G}}$, giving us the pullback diagram
\[\begin{tikzcd}
E\mathscr{G} \times_{\mathscr{G}} M \ar[r] \ar[d] &
E\overline{\mathscr{G}} \times_{\overline{\mathscr{G}}} M \ar[d] \\
B\mathscr{G} \ar[r] & B\overline{\mathscr{G}}
\end{tikzcd}\]
The map $E\mathscr{G} \times_{\mathscr{G}} M
\to E\overline{\mathscr{G}} \times_{\overline{\mathscr{G}}} M$ is a
$B \mathrm{U}_1$-bundle, giving us the diagram
\[\begin{tikzcd}
B \mathrm{U}_1 \ar[r] \ar[d,equals]& E\mathscr{G} \times_{\mathscr{G}} M \ar[r] \ar[d] &
E\overline{\mathscr{G}} \times_{\overline{\mathscr{G}}} M \ar[d] \\
B \mathrm{U}_1 \ar[r] & B\mathscr{G} \ar[r] & B\overline{\mathscr{G}}
\end{tikzcd}\]
From this, we can deduce that the map
$B \mathrm{U}_1 \to E\mathscr{G}\times_{\mathscr{G}} M$ induces a surjection
on rational cohomology, so we can apply Leray-Hirsch to the bundle
$E\mathscr{G} \times_{\mathscr{G}} M
\to E\overline{\mathscr{G}} \times_{\overline{\mathscr{G}}} M$,
giving us
\[
H^\bullet_{\mathscr{G}}(M,\Q) \cong
H^\bullet(B\mathrm{U}_1, \Q) \otimes H^\bullet_{\overline{\mathscr{G}}}(M,\Q)
\]
In terms of Poincar\'e series, we have
\[
P_{t,\mathscr{G}}(M) = \frac{P_{t,\overline{\mathscr{G}}}(M)}{1-t^2}
\]
In our specific case, letting $M = \mathscr{C}_{ss}(E)$, we get
\[
P_t(N(n,k)) = (1-t^2)P_{t,\mathscr{G}}(\mathscr{C}_{ss}(E))
\]
In theory, this gives us all the results we need to compute the Ponicar\'e
series for $N(n,k)$. However, it is not immediately clear how the pieces
fit together. To get a better idea, we will worth through the case
$n = 2$ and $k = 1$. We first take inventory of the facts and formulas we need.
\begin{enumerate}
  \item The Poincar\'e polynomial for the classifying space of the gauge group is
  \[
  P_t(B\mathscr{G}) = \frac{(1+t)^{2g}(1+t^3)^{2g}}{(1-t^4)(1-t^2)^2}
  \]
  \item  Let $k_\mu$ denote the real codimension of the strata $\mathscr{C}_\mu(E)$
  inside of $\mathscr{C}(E)$. Then
  \[
  P_t(B\mathscr{G}) = \sum_{\mu}t^{k_\mu}P_{\mathscr{G}}(\mathscr{C}_\mu(E))
  \]
  \item Let the $D_i$ be the succesive quotients coming from the Harder-Narasimhan
  filtration of $E$. Then
  \[
  P_{t,\mathscr{G}}(\mathscr{C}_\mu(E))
  = \prod_iP_{t,\mathscr{G}(D_i)}(\mathscr{C}_{ss}(D_i))
  \]
  \item Let $n_i = \dim D_i$ and $k_i = \deg(D_i)$. Then the codimension
  of the strata $\mathscr{C}_\mu(E)$ is given by
  \[
  2\sum_{i > j}n_ik_j-k_in)j + n_in_j(g-1)
  \]
\end{enumerate}
%
We now identify the possible Harder-Narasimhan types for a holomorphic strcture
on $E$. If $E$ is a semistable bundle, then its Harder-Narasimhan filtration
is just $0 \subset E$, and the Harder-Narasimhan type is $(1/2,1/2)$. Otherwise,
there exists an rank one subbundle $L \subset E$ with $\mu(L) > \mu(E) = 1/2$, and
the Harder-Narasimhan filration is $0 \subset L \subset E$. This means that the
Harder-Narasimhan type of $E$ is entirely determined by the degree of $L$, since
we can recover the degree of $E/L$ as $1 - \mathrm{deg}(L)$, so the
Harder-Narasimhan type would be $(\mathrm{deg}(L), 1- \mathrm{deg}(L))$. For
notational convenience, let $\mathscr{C}_r(E)$ denote the stratum corresponding
to the type $(r+1,-r)$. Then we have
\[
P_{t,\mathscr{G}}(\mathscr{C}_r(E))
= P_{t,\mathscr{G}(L)}(\mathscr{C}_{ss}(L)) P_{t,\mathscr{G}(E/L)}(\mathscr{C}_{ss}(E/L))
\]
We note that both $L$ and $E/L$ are both line bundles, which are automatically stable,
so $\mathscr{C}_{ss}(L) = \mathscr{C}(L)$ and $\mathscr{C}_{ss}(E/L) = \mathscr{C}(E/L)$.
Furthmore, our formula for the Poincar\'e series for the classifying space
for the gauge group of a line bundle gives us
\[
P_t(B\mathscr{G}(L)) = P_t(B\mathscr{G}(E/L)) = \frac{(1+t)^{2g}}{1-t^2}
\]
Therefore, we get
\[
P_{t, \mathscr{G}}(\mathscr{C}_r(E)) = \left(\frac{(1+t)^{2g}}{1-t^2}\right)^2
\]
We now need to compute the codimensions $k_r$ of the strata $\mathscr{C}_r(E)$.
Using the formula we derived earlier, we have
\[
k_r = 4r + 2g
\]
Putting everything together, we get the following identity
\[
\frac{(1+t)^{2g}(1+t^3)^{2g}}{(1-t^4)(1-t^2)^2} = P_{t,\mathscr{G}}(\mathscr{C}_{ss}(E))
+ \sum_{r=0}^\infty t^{4r+2g}\left(\frac{(1+t)^{2g}}{1-t^2}\right)^2
\]
After some manipulations and rearranging, this becomes
\[
P_{t, \mathscr{G}}(\mathscr{C}_{ss})(E)
= \frac{(1+t)^{2g}(1+t^3)^{2g}-t^{2g}(1+t)^{4g}}{(1-t^4)(1-t^2)^2}
\]
Finally, using the relationship between $\mathscr{G}$-equivariant cohomology
and $\overline{\mathscr{G}}$-equivariant cohomology, we get
\begin{align*}
P_t(N(2,1)) &= (1-t^2)P_{t,\mathscr{G}}(\mathscr{C}_{ss}(E)) \\
&= \frac{(1+t)^{2g}(1+t^3)^{2g} - t^{2g}(1-t)^{4g}}{(1-t^4)(1-t^2)}
\end{align*}
%
%
\newpage
%
\nocite{*}
%
\printbibliography
%
\end{document}