%
\section{Equivariant Cohomology}
%
References for this material include \cite{TuEC} and \cite{Kirwan}. \\

Our goal now is to compute the cohomology of the moduli space
of semistable holomorphic bundles over $X$ of rank $n$ and degree $k$,
denoted $N(n,k)$, in the case that $n$ and $k$ are coprime. Recall that
$N(n,k)$ has a global quotient description as
$N(n,k) = \mathscr{C}_{ss}(E)/\mathscr{G}_\C$. The computation of
the cohomology of $N(n,k)$ uses the concept of \ib{equivariant cohomology},
which is a cohomology theory for spaces with the action of a group. Before
doing so, we give a bit of exposition on where equivariant cohomology fits
into our story. \\

Recall that we mentioned that a $\mathrm{U}_n$-Yang-Mills connection is
equivalent to the choice of some Hermitian matrix, with some conditions
on the eigenvalues. Using the correspondence between unitary connections
and holomorphic structures, these conditions actually reflect the
Harder-Narasimhan filtration of the corresponding holomorphic bundle.
Using the perspective of unitary connections, Atiyah and Bott managed to show that
the Harder-Narasimhan stratification is \ib{equivariantly perfect},
i.e. looks like it came from an equivariantly perfect Morse function. In
other words, we can understand the equivariant cohomology of $\mathscr{C}(E)$
as being built from a simple formula involving the equivariant cohomology
of the strata $\mathscr{C}_\mu(E)$. This gives us a formula
for the equivariant cohomology of the semistable strata, from which
we can compute the ordinary cohomology of the quotient $N(n,k)$.
%
\begin{defn}
Let $G$ be a Lie group. A \ib{classifying space} for $G$ is a topological
space $BG$ such that we have a functorial correspondence
\[
\set{\text{Principal }G\text{-bundles } P \to X} \longleftrightarrow
\set{\text{Homotopy classes of maps }X \to BG}
\]
\end{defn}
%
Suppose $EG$ is a contractible space with a free action of $G$. Then the quotient
map $EG \to EG/G$ is a principal $G$-bundle.
%
\begin{thm}
The quotient space $BG = EG/G$ is a classifying space for $G$, which assigns to
a homotopy class of maps $[f] : X \to BG$ the principal $G$-bundle $f^*EG$,
which is independent of our choice of representative of $f$. Furthermore,
The spaces $EG$ and $EG/G$ are unique up to homotopy equivalence.
\end{thm}
%
Since the spaces $EG$ and $BG$ are only well-defined up to homotopy equivalence,
we will call a specific choice of $EG$ and $BG$ a \ib{model} for $EG \to BG$.
%
\begin{defn}
Let $G$ be a Lie group and $X$ a smooth manfold with an action of $G$. The
\ib{equivariant cohomology} of $X$, denoted $H^\bullet_G(X,\Z)$, is the
ordinary cohomology of the total space of the associated bundle
$EG\times_G X \to BG$, i.e.
\[
H^\bullet_G(X,\Z) \defeq H^\bullet(EG\times_G X,\Z)
\]
\end{defn}
%
We note that since any two models for $EG \to BG$ are homotopy equivalent,
this is well defined. In addition, we can define the equivariant cohomology over
any other coefficient group we want. Intuitively, the equivariant cohomology of
$X$ is is something like the regular cohomology of the quotient space $X/G$, but
this isn'texactly true when the action isn't nice. The difference is essentially
that the equivariant cohomology keeps track of stabilizer subgroups. To
see this, note that we have a map $EG\times_G X \to X/G$ taking an
equivalence class $[p,x] \in EG\times_G X$ to the orbit $[x] \in X/G$.
%
\begin{prop}
The fiber of $EG\times_G X \to X/G$ over $[x]$ is the quotient space $EG/G_x$, where
$G_x$ is the stabilizer of any representative of $[x]$.
\end{prop}
%
\begin{proof}
The fiber over $[x]$ is the subset $\set{[p,x\cdot g] ~:~ g \in G}\subset EG \times_G X$.
equivalently, this is the subset $\set{[p\cdot g\inv,x] ~:~ g \in G}$. Using
representatives of this form then gives the desired result.
\end{proof}
%
We note that since $EG$ is contractible and the action of $G_x \subset G$
is free on $EG$, the fiber over $[x]$ is a model for $BG_x$.
%
\begin{cor}
If the action of $G$ on $X$ is free, then the map $EG\times_G X \to X/G$
is a homotopy equivalence.
\end{cor}
%
\begin{cor}
If $X$ is contractible, then $H^\bullet_G(X,\Z) = H^\bullet(BG,\Z)$.
\end{cor}
%
\begin{proof}
The map $EG\times_G X \to X/G$ is a fiber bundle with contractible fiber $EG$.
\end{proof}
%
We now return to the computation of the cohomology of $N(n,k)$. The heavy
lifting for this computation comes from the following theorems:
%
\begin{thm}
Let $E \to X$ be a holomorphic vector bundle with Harder-Narasimhan filtration
\[
0 = E_0 \subset E_1 \subset \cdots \subset E_r = E
\]
Choose a $C^\infty$ splitting of the Harder-Narasimhan filtration, giving us a direct
sum decomposition as $C^\infty$ bundles
\[
E = \bigoplus_i D_i
\]
such that
\[
E_i = \bigoplus_{j < i} D_j
\]
Then we have
\[
H^\bullet_{\mathscr{G}_\C}(\mathscr{C}_\mu(E),\Q) \cong
\bigotimes_{i=1}^r H^\bullet_{\mathscr{G}_\C(D_i)}(\mathscr{C}_{ss}(D_i),\Q)
\]
where $\mathscr{G}_\C(D_i)$ denotes the group of smoot bundle automorphisms of $D_i$.
\end{thm}
%
\begin{thm}
Let $k_\mu$ denote the real codimension of $\mathscr{C}_\mu(E)$ inside
of $\mathscr{C}(E)$, and let $P_{t,\mathscr{G}_\C}(X)$ denote the
$\mathscr{G}_\C$-equivariant Poincar\'e series for a space $X$, i.e.
\[
P_{t,\mathscr{G}_\C}(X) \defeq \sum_i (\dim H^\bullet_{\mathscr{G}_\C}(X,\Q))t^i
\]
Then we have
\[
P_{t,\mathscr{G}_\C}(\mathscr{C}(E))
= \sum_\mu t^{k_\mu}P_{t,\mathscr{G}_\C}(\mathscr{C}_\mu(E))
\]
\end{thm}
%
\begin{thm}
The Poincar\'e seires for $B\mathscr{G}_\C$ is
\[
P_t(\mathscr{G}_\C) = \frac{\prod^n_{k=1}(1+t^{2k-1})^2g}
{(1-t^2n)\prod_{k=1}^{n-1}(1-t^{2k})^2}
\]
where $g$ is the genus of $X$.
\end{thm}
%
The first theorem tell us that we can understand the $\mathscr{G}_\C$-equivariant
cohomology of the strata by understanding the $\mathscr{G}_\C$-equivariant
cohomology of the semistable strata for lower dimensional bundles,
which will give us an inductive procedure for computing the Poincar\'e series.
The second theorem tells us that the equivariant cohomology of
the entire space is a simple expression in terms of the equivariant cohomology
of the strata. Since $\mathscr{C}(E)$ is contractible, the third
theorem tells us that if we can compute the equivariant cohomology of
all the strata except for the semistable locus, then we can compute the
equivariant cohomology of the semistable strata. \\

We first compute the codimension of the strata $\mathscr{C}_\mu(E)$. To do
this, we will use some facts regarding infinitesimal variations of holomorphic
structures.
%
\begin{prop}
Let $E \to X$ be a holomorphic vector bundle. The infinitesimal variations
of the holomorphic structure on $E$ are given by the Dolbeault cohomology group
$H^1_{\dbar}(X, \End(E))$.
\end{prop}
%
We are being purposefully vague when we say ``infinitesimal variation of holomorphic
structure." Our main use for the result is to identify the normal directions
of the strata $\mathscr{C}_\mu(E)$. The isomorphism class of a holomorphic
structure on a $C^\infty$ vector bundle $E$ is given by a $\mathscr{G}_\C$-orbit
in $\mathscr{C}(E)$, and the infinitesimal variation can be interpreted as the
normal directions to this orbit. This gives us a way to compute the
codimension of the strata $\mathscr{C}_\mu(E)$. From this perspective,
the normal directions to $\mathscr{C}_\mu(E)$ consist of infinitesimal variations
that change the type of the Harder-Narasimhan filtration. Explicitly,
we have a holomorphic subbundle $\End'(E) \to \End(E)$ consisting of
holomorphic endomorphisms of $E$ that preserve the Harder-Narasimhan filtration.
Then we can identify $H^1_{\dbar}(X,\End'(E))$ with the infinitesimal varations
consisting of the directions tangent to $\mathscr{C}_\mu(E)$. Furthermore, if
we let $\End"(E)$ denote the quotient bundle $\End(E)/\End'(E)$,
we can identify $\End"(E)$ with the holomorphic bundle endomorphism that do
not preserve the Harder-Narasimhan filtration, which tells us that the complex
codimension of $\mathscr{C}_\mu(E)$ in $\mathscr{C}(E)$ is the dimension of
$H^1_{\dbar}(X,\End"(E))$. To compute this, we use Riemann-Roch.
%
\begin{thm}[\ib{Riemann-Roch}]
Let $E \to X$ be a holomorphic vector bundle, where $X$ is genus $g$, and let
$h^i(E) = \dim H^i_{\dbar}(X,E)$. Then
\[
h^0(E) - h^1(E) = c_1(E) + (1-g)\mathrm{rank}(E)
\]
\end{thm}
%
Because of Riemann-Roch, it suffices to compute the dimension of
$H^0_{\dbar}(X,E)$ to compute $H^1_{\dbar}(X,E)$, and we want to apply
this to the holomorphic bundle $\End"(E)$.
%
\begin{prop}
\[
H^0_{\dbar}(X,\End"(E)) = 0
\]
\end{prop}
%
\begin{proof}
An element $g \in H^0_{\dbar}(X,\End"(E))$ is a global holomorphic endomorphism of
$E$ that does not fix the Harder-Narasimhan filtration. By assuption,
there exists some subbundle $E_i$ with $i > 0$ in the filtration such that
$g(E_i) \not\subset E_i$. By minimality of $i$, we have that
$g(E_{i-1}) \subset E_{i-1}$. Then let $k$ be the smallest integer such
that $g(E_i) \subset E_k$. Then the restriction of $g$ to $E_k$ factors
through the quotients to a nontrivial bundle homomorphism
$E_i/E_{i-1} \to E_k/E_{k-1}$. We note that both $E_i/E_{i-1}$ and $E_k/E_{k-1}$
are semistable and satisfy $\mu(E_i/E_{i-1}) > \mu(E_k/E_{k-1})$ by the properties
of the Harder-Narasimhan filtration. Let $K \subset E_i/E_{i-1}$ be
the smallest holomorphic subbundle containing the kernel, and
$A \subset E_k/E_{k-1}$ the smallest holomorphic subbundle containing the image,
giving us the short exact sequence of holomorphic bundles
\[\begin{tikzcd}
0 \ar[r] & K \ar[r] & E_i/E_{i-1} \ar[r] & A \ar[r] & 0
\end{tikzcd}\]
Semistability of $E_i/E_{i-1}$ implies that $\mu(A) \leq \mu(E_k/E_{k-1})$,
so $\mu(A) < \mu(E_i/E_{i-1})$. However, semistability of $E_i/E_{i-1}$ also
implies that $\mu(K) \leq \mu(E_i/E_{i-1})$, which would imply that
$\mu(E_i/E_{i-1}) \leq \mu(A)$, a contradiction.
\end{proof}
%
To use Riemann-Roch, we must identify the rank and degree of $\End"(E)$.
Since both of these quantities are topological invariants, we may work
in the $C^\infty$ category. We first compute the degree.
Since $\End(E) \cong E^*\otimes E$, we have that
$\mathrm{deg}(E) = 0$, where we use the fact that the degree of a bundle
is the same as the degree of its determinant line, and the formula for
the determinant line of a tensor product of bundles. Then since the
degree is additive in exact sequences, we get
\[
\mathrm{deg}(\End'(E)) + \mathrm{deg}(\End"(E)) = 0
\]
We then compute $\mathrm{deg}(\End'(E))$, which will tell us $\mathrm{deg}(\End"(E))$.
Fix a smooth splitting of the Harder-Narasimhan filtration, giving us a
$C^\infty$ decomposition
\[
E = \bigoplus_i D_i
\]
This gives us the identification as smooth bundles
\[
\End'(E) = \bigoplus_{i \geq j}\hom(D_i,D_j)
\]
Then if we let $\mu(D_i) = k_i/n)i$, we get
\[
\mathrm{deg}(\End'(E)) = \sum_{i\geq j} k_jn_i - k_in_j
\]
where we use additivity of degree with respect to direct sums and
the identification of $\hom(D_i,D_j)$ with $D_i^*\otimes D_j$. In the
case $i = j$, we get $\hom(D_i,D_j) = \End(D_i)$, which has degree $0$,
so we get
\[
\mathrm{deg}(\End'(E)) = \sum_{i > j} k_jn_i - k_in_j
\]
Negating this gives the degree of $\End"(E)$. \\

For the rank, this comes easily from the $C^\infty$ decomposition
\[
\End"(E) = \bigoplus_{i > j}\hom(D_i,D_j)
\]
giving us
\[
\mathrm{rank}(\End"(E)) = \sum_{i < j} n_in_j
\]
Putting everything together gives us
\[
\dim(H^1_{\dbar}(X,\End"(E))) = \sum_{i > j}n_ik_j-k_in)j + n_in_j(g-1)
\]
which by our earlier discussion, is the complex codimension of the strata
$\mathscr{C}_\mu(E)$.
%