%
\section{Holomorphic Vector Bundles and Yang-Mills Connections}
%
A good references for this material would be \cite{Kob} and \cite{McDuff}\\

Let $X$ denote a complex manifold.
%
\begin{defn}
A \ib{holomorphic vector bundle} is a complex vector bundle $\pi : E \to X$
such that the total space $E$ is a complex manifold and $\pi$ is holomorphic.
\end{defn}
%
Given a holomorphic vector bundle $E \to X$, we can find a trivialization
of $E$ such that the transition functions are holomorphic. In a neighborhood
$U \subset X$ such that $E\vert_U$ is trivial, the smooth sections can
be identified with functions $U \to \C^n$, and the holomorphic sections
can be identified with the holomorphic functions $U \to \C^n$. We
have a local operator $\dbar$, which we can apply componentwise to a local
section to get an operator on smooth sections over $U$. Furthermore, since
$\dbar$ annihilates holomorphic functions and the transition functions
are holomorphic, we have that $\dbar$ glues to a well defined operator
$\dbar_E : \mathcal{A}^0_X(E) \to \mathcal{A}^{0,1}_X(E)$. The holomorphic sections of $E$
are then exactly the sections annihilated by $\dbar_E$. Furthermore, the operator
$\dbar_E$ extends to operators $\dbar_E : \mathcal{A}^k_X(E) \to \mathcal{A}^{k+1}_X(E)$,
and satifies the condition $\dbar_E^2 = 0$, since $\dbar^2 = 0$. The punchline
is that the holomorphic structure on $E$ is entirely determined by this operator.
%
\begin{thm}
Let $\pi : E \to X$ be a $C^\infty$ complex vector bundle, and let
$D : \mathcal{A}^0_X(E) \to \mathcal{A}^{0,1}(E)$ be an operator satifying
$D^2 = 0$. Then there exists a unique complex structure on $E$
such that $\pi$ is holomorphic and $D$ coincides with the operator $\dbar_E$.
\end{thm}
%
This can be seen as a linearized version of the Newlander-Nirenberg theorem.
In particular, a holomorphic vector bundle $E \to X$ can be thought
of as a smooth vector bundle along with a choice of operator $\dbar_E$.
Since the operator $\dbar$ satisfies a Leibniz rule, the operator $\dbar_E$
behaves like a connection. In a \emph{smooth} local trivialization,
we can write
\[
\dbar_E = \dbar + B
\]
where $B$ is a smooth $\mathrm{M}_n\C$-valued $(0,1)$-form. Indeed,
we have that the space of holomorphic structures on a smooth vector bundle
$E \to X$ is an affine space over $\mathcal{A}^{1,0}(\End E)$. We
let $\mathscr{C}(E)$ denote the space of holomorphic structures on $E$.\\

We now restrict to the case where $X$ is a Riemann surface.
%
\begin{defn}
The \ib{slope} of a holomorphic vector bundle $E \to M$ is
\[
\mu(E) \defeq \frac{c_1(E)}{\mathrm{rank}(E)}
\]
where we think of $c_1(E) \in H^2(X,\Z)$ as an integer via integration over $X$.
\end{defn}
%
Sometimes the integer $c_1(E)$ is also referred to as the \ib{degree} of $E$.
One thing to note is that the slope of a holomorphic vector bundle is
independent of the holomorphic structure -- both the degree and rank are
topological invariants, and only depend on the underlying $C^\infty$ complex
vector bundle.
%
\begin{defn}
A holomorphic vector bundle $E \to X$ is
\begin{enumerate}
  \item \ib{Stable} if for every holomorphic subbundle $F \subset E$,
  we have $\mu(F) < \mu(E)$.
  \item \ib{Semistable} if for every holomorphic subbundle $F \subset E$,
  we have $\mu(F) \leq \mu(E)$.
  \item \ib{Unstable} if $E$ is not semistable.
\end{enumerate}
\end{defn}
%
While the slope is a topological invariant, stability is not, since
we only consider holomorphic subbundles -- which depend on the holomorphic structure.
The terminology comes from Geometric Invariant Theory (GIT). The main
result will use is:
%
\begin{thm}[\ib{The Harder-Narasimhan Filtration}]
Let $E \to X$ be a holomorphic vector bundle. Then $E$ admits a canonical filtration
\[
0 = E_0 \subset E_1 \subset \cdots \subset E_n = E
\]
by holomorphic subbundles $E_i$ such that $E_i/E_{i-1}$ is semistable and
\[
\mu(E_1/E_0) > \mu(E_2/E_1) > \cdots > \mu(E_n/E_{n-1})
\]
\end{thm}
%
The proof of the above theorem is not extremely difficult, but we omit it.
The main idea is that any holomorphic vector bundle has a unique
maximal semistable subbundle, which we take to be $E_1$. We then take
$E_2$ to be the preimage of the maximal semistable bundle of $E_1/E_0$ under
the quotient map, and continue inductively. The slopes
$\mu_i \defeq \mu(E_i/E_{i-1})$ gives us $n$ rational numbers. If $k$ denotes
the rank of $E$, then we construct an element of $\Q^k$ by arranging
the $\mu_i$ in order, and repeating the entry $\mu_i$ a total of
$\mathrm{rank}(E_i/E_{i-1})$ times. We call this vector the
\ib{Harder-Narasimhan type} of $E$. \\

Our ultimate goal will be to relate moduli spaces of holomorphic vector
bundles over $X$ to Yang-Mills connections. To see this, let $E \to X$ be
a $C^\infty$ complex vector bundle of rank $n$, and fix a Hermitian metric
on $E$. Then let $P \to X$ denote the principal $\mathrm{U}_n$-bundle of frames
for $E$. We abbreviate the gauge group $\mathscr{G}(P)$ as $\mathscr{G}$.
%
\begin{prop}
There is a bijection $\mathscr{A}(P) \leftrightarrow \mathscr{C}(E)$.
\end{prop}
%
\begin{proof}
We provide maps in both directions. Suppose we have a connection
$A \in \mathscr{A}(P)$. Then $\mathcal{A}$ induces a covariant derivative
$d_A : \mathcal{A}^0_X(E) \to \mathcal{A}^1_X(E)$. The $(0,1)$ part of $d_A$
automatically satisfies $(d_A^{0,1})^2 = 0$, since $\mathcal{A}^2_X = 0$ by
dimension reasons. Therefore, $d_A^{0,1}$ defines a holomorphic structure
on $E$. \\

In the other direction, given a holomorphic structure $\dbar_E$,
there exists a unique Hermitian connection $A$ such that $d_A^{1,0} = \dbar_E$
called the \ib{Chern connection}, which is a sort of analogue to the Levi-Civita
connection in Riemannian geometry.
\end{proof}
%
Let $\mathscr{G}_\C$ denote the group of smooth bundle automorphisms of $E$.
Though both $\mathscr{G}_\C$ and $\mathscr{G}$ are both infinite dimensional,
the former can be seen as the complexification of the latter. The space
$\mathscr{C}(E)$ has a natural action by $\mathscr{G}_\C$ by conjugation.
Furthermore, the orbits under this action are exactly the isomorphism
classes of holomorphic structures on $E$. This is most easily seen by
characterizing an isomorphism $\varphi : E \to F$ of holomorphic vector bundles
as a smooth bundle isomorphism intertwining $\dbar_E$ and $\dbar_F$. However,
the na\"ive quotient $\mathscr{C}(E)/\mathscr{G}_\C$ is poorly behaved
(for example, it is not Hausdorff). To remedy this, we restrict our attention
to semistable bundles. \\

The relationship between $\mathscr{G}_\C$ and $\mathscr{G}$ as well
as the identification of $\mathscr{A}(P)$ and $\mathscr{C}(E)$ suggests
that isomorphism classes of holomorphic bundles should have something to
do with gauge equivalence classes of connections on $P$. This is turns out
to be true, and is an infinite dimensional version of the relationship between
a GIT quotient and a symplectic quotient. To investigate further, we make a short
digression regarding this relationship. \\

Let $G$ be a reductive complex group, and $X$ a K\"ahler manifold with K\"ahler
metric $\omega$, equipped with a ``nice" action of $G$. In the usual setting,
$X$ is a smooth projective variety with a fixed embedding $X \hookrightarrow \CP^N$,
the K\"ahler metric $\omega$ is the restriction of the Fubini-Study form,
and the $G$-action is induced by a homomorphism $G \to \GL_{N+1}(\C)$. In general,
the na\"ive quotient $X/G$ is not well behaved, and one restricts the action to a
subset $X_{ss}$ consisting of \ib{semistable points} to construct the
\ib{GIT quotient} $X_{ss}/G$. \\

Then let $K \subset G$ denote the maximal compact subgroup, which has the
property that its complexification is isomorphic to $G$. Suppose that
the action of $K$ on $X$ is symplectic, i.e. the action of any $k \in K$
preserves the K\"ahler metric on $X$. Let $\mathfrak{k}$ denote the
Lie algebra of $K$. Then the infinitesimal action of $K$ is given by
the Lie algebra homomorphism $\mathfrak{k} \to \mathfrak{X}(X)$ (where
$\mathfrak{X}(X)$ denotes the space of vector fields on $X$) defined by
$\xi \mapsto X_\xi$ where
\[
(X_\xi)_p \defeq \frac{d}{dt}\bigg\vert_{t=0}\mathrm{exp}(t\xi)\cdot p
\]
\begin{defn}
A symplectic action of $K$ on $X$ is \ib{Hamiltonian} if for each
$\xi \in \mathfrak{k}$, there exists a function $H_\xi : X \to \R$
such that for all $p \in X$ and $v \in T_pX$, we have
\[
\omega_p((X_\xi)_p,v) = (dH_\xi)_p(v)
\]
and the mapping $\xi \mapsto H_\xi$ is $K$-equivariant with respect
to the right actions of $K$ on $\mathfrak{k}$ by the Adjoint action
and precomposition with left translation $L_k$ on $C^\infty(X)$. The
functions $H_\xi$ are called \ib{Hamiltonian functions}.
\end{defn}
%
\begin{defn}
Suppose we have a Hamiltonian action of $K$ on $X$. A \ib{moment map}
for the action is a $K$-equivariant map $X \to \mathfrak{k}^*$ (where the
action on $\mathfrak{k}$ is the coadjoint action) such
that for any $p \in X$, $v \in T_pX$, and $\xi \in \mathfrak{k}$, we have
\[
d\mu_p(v)(\xi) = \omega_p((X_\xi)_p,v)
\]
\end{defn}
%
One things to note is that the Hamiltonian functions can be
recovered by the moment maps. If a Hamiltonian action admits a moment map,
then
\[
H_\xi(p) = \mu(p)(\xi)
\]
The let $\langle\cdot,\cdot\rangle$ be an inner product on $\mathfrak{k}^*$
that is invariant under the coadjoint action, and $\norm{\cdot}$ the induced
norm. Since $X$ is compact, the map $\norm{\mu}^2 : X \to \R$ attains
its minimum, and WLOG we assume that the minimum value is $0$.
%
\begin{defn}
The \ib{symplectic quotient} of $X$ by $K$ is the quotient space
\[
\mu\inv(0)/K
\]
\end{defn}
%
The symplectic quotient can also be referred to as the \ib{symplectic reduction}
or the \ib{Marsden-Weinstein quotient}.
%
\begin{thm}
The symplectic quotient of $X$ by $K$ admits a unique K\"ahler structure
such that the K\"ahler metric on $\mu\inv(0)/K$ is induced by the K\"ahler
metric on $X$.
\end{thm}
%
The relationship between the GIT quotient and the symplectic quotient is
given by the Kempf-Ness theorem.
%
\begin{thm}[\ib{Kempf-Ness}]
Suppose a complex reductive group $G$ acts on a K\"ahler manifold $X$
such that the action of the maximal compact subgroup $K \subset G$ is
Hamiltonian and admits a moment map $\mu : X \to \mathfrak{k}^*$. Then
the $G$-orbit of any semistable point contains a unique $K$-orbit minimizing
$\norm{\mu}^2$. This establishes a homeomorphism
\[
X_{ss}/G \longleftrightarrow \mu\inv(0)/K
\]
\end{thm}
%
We now want to relate the previous discussion to our situation. Using
the identification of $\mathscr{A}(P)$ and $\mathscr{C}(E)$, we
want the action of $\mathscr{G}_\C$ to play the role of the complex reductive
group $G$ and the gauge group $\mathscr{G}$ to play the role of the maximal
compact subgroup. Since the space $\mathscr{A}(P)$ is infinite dimensional,
along with the groups $\mathscr{G}_\C$ and $\mathscr{G}$, we are working
in an infinite dimensional setting, but we will gloss over the analytic
details and work with them formally. \\

Our first task is to realize $\mathscr{A}(P)$ as a ``K\"ahler manifold."
Since $X$ is a surface, the Hodge star maps $\mathcal{A}^1_X(\g_P)$ to
itself and squares to $-1$, so it defines a ``complex structure" on
$\mathscr{A}(P)$, where we use the fact that $\mathscr{A}(P)$ is affine
over the vector space $\mathcal{A}^1_X(\g_P)$ to identify the ``tangent space"
of $\mathscr{A}(P)$ at a connection $A$ with $\mathcal{A}^1_X(\g_P)$.
Furthermore, the fact that for $1$-forms $\omega,\eta \in \mathcal{A}^1_X(\g_P)$
the pairing $\langle\omega\wedge\eta\rangle$ is skew-symmetric, we
can identify the pairing
\[
\omega \otimes \eta \mapsto \int_X \langle\omega\wedge\eta\rangle
\]
as a ``symplectic form" on $\mathscr{A}(P)$. Together, these give
$\mathscr{A}(P)$ the structure of a ``K\"ahler manifold." \\

Our next task is to show that the action of $\mathscr{G}$ on $\mathscr{A}(P)$ is
``Hamiltonian" with respect to this K\"ahler structure. One can
identify the ``Lie algebra" of $\mathscr{G}$ with the space of
sections $\Gamma(X,\g_P)$.
%
\begin{prop}
The infinitesimal action of $\phi \in \Gamma(X,\g_P)$ on $\mathscr{A}(P)$ is
given by the mapping $A \mapsto d_A\phi$.
\end{prop}
%
\begin{proof}
We compute the vector field at a connection $A \in \mathscr{A}(P)$ to be
\begin{align*}
&\frac{d}{dt}\bigg\vert_{t=0} \Ad_{\exp(t\phi)\inv} A + \exp(t\phi)^*\theta
= -[\phi, A] + \frac{d}{dt}\bigg\vert_{t=0}(dL_{\exp(-t\phi)} d(\exp(t\phi))) \\
&= [A,\phi] + \left(\frac{d}{dt}\bigg\vert_{t=0} dL_{\exp(-t\phi)}\right)d(\exp(0))
+ dL_{\exp(0)}\left(\frac{d}{dt}\bigg\vert_{t=0}d(\exp(t\phi))\right) \\[5pt]
&= [A,\phi] + d\phi\\
&= d_A\phi
\end{align*}
where for the third equality we use the product rule, and in the fourth equality
we use the fact that $\exp(0) = \id$ and that the derivative of
$\exp(t\phi)$ as $t \to 0$ is $\phi$.
\end{proof}
%
\begin{prop}
Let $\phi \in \Gamma(X,\g_P)$. Then the function
\begin{align*}
H_\phi : \mathscr{A}(P) &\to \R \\
A &\mapsto \int_X\langle F_A\wedge\phi\rangle
\end{align*}
is a Hamiltonian function for $\phi$.
\end{prop}
%
\begin{exer}
Prove the previous proposition.
\end{exer}
%
Since $\langle\cdot,\cdot\rangle$ is invariant under the adjoint action,
the mapping $\phi \mapsto H_\phi$ is clearly $\mathscr{G}$ equivariant,
so this tells us that the action is Hamiltonian. Furthermore,
the computation we made identifies the mapping $A \mapsto F_A$ as the
moment map for this action. To summarize, we have the following
analogies
%
\begin{align*}
\text{K\"ahler manifold }X &\longleftrightarrow \mathscr{A}(P) \\
\text{Complex reductive group }G &\longleftrightarrow \mathscr{G}_\C \\
\text{Maximal compact subgroup }K \subset G &\longleftrightarrow \mathscr{G} \\
\text{Moment map }\mu &\longleftrightarrow A \mapsto F_A \\
\text{Norm square of the moment map} \norm{\mu}^2 &\longleftrightarrow L
\end{align*}
%
The last missing piece is something analogous to the Kempf-Ness theorem.
%
\begin{thm}[\ib{Narasimhan-Seshadri}]
Let $\mathscr{A}_s(P) \subset \mathscr{A}(P)$ denote the subspace of connections
that are absolute minimal for the Yang-Mills functional, and correspond
to irreducible representations $\Gamma_\R \to \mathrm{U}_n$. Let $\mathscr{C}_s(E)$
denote the subspace of stable holomorphic structures on $E$. The isomorphism
classes of holomorphic bundles in $\mathscr{C}_s(E)$ admit unique
Yang-Mills connections (up to gauge equivalence) minimizing the Yang-Mills functional.
In other words, there is a homeomorphism
\[
\mathscr{A}_s(P)/\mathscr{G} \longleftrightarrow \mathscr{C}_s(E)/\mathscr{G}_\C
\]
\end{thm}
%
\begin{rem*}
The original proof is more algebraic in flavor. A proof more in the spirit
of the Atiyah-Bott paper was given by Donaldson in \cite{donaldson1983}.
The spirit of this proof is carried on by the proof of Hermitian-Yang-Mills
and the nonabelian Hodge theorem, which were both grew out of the
developments from the Atiyah-Bott paper.
\end{rem*}
%
One issue is that the Narasimhan-Seshadri theorem only works for
stable bundles. However, in the case that the rank and degree of $E$
are coprime, stability and semistability coincide for numerical reasons.
For that reason, we will continue onwards with the assumption that the
rank and degree of $E$ are coprime.
%