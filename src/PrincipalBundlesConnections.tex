%
\section{Principal Bundles and Connections}
%
Fix a compact manifold $X$ and a Lie group $G$.
%
\begin{defn}
A \ib{principal $G$-bundle} is a fiber bundle $\pi : P \to M$ with a smooth
right $G$ action such that:
\begin{enumerate}
  \item The action of $G$ preserves the fibers of $\pi$, and gives each fiber
  $P_x \defeq \pi\inv(x)$ the structure of a \ib{right $G$-torsor}, i.e. the
  action of $G$ on $P_x$ is free and transitive.
  \item For every point $x \in X$, there exists a \ib{local trivialization} of
  $P$, i.e. a diffeomorphism $\varphi : P\vert_U \defeq \pi\inv(U) \to U \times G$
  that is $G$-equivariant (where the action on $U \times G$ is right multiplication
  with the second factor) and the following diagram commutes:
  \[\begin{tikzcd}
  P\vert_U \ar[rr, "\varphi"] \ar[dr, "\pi"'] && U \times G \ar[dl] \\
  & U
  \end{tikzcd}\]
  where the map $U\times G \to U$ is projection onto the first factor.
\end{enumerate}
\end{defn}
%
\begin{exmp}
Let $E \to X$ be a real vector bundle of rank $k$. For $x \in X$, let $\B_x$
denote the set of all bases of the fiber $E_x$, i.e. the set of linear isomorphisms
$\R^k \to E_x$. This has a natural right action of $\GL_k\R$ by precomposition.
Furthermore, this action is free and transitive, giving $\B_x$ the structure
of a $\GL_k\R$-torsor. Then let
\[
\B_{\GL_k\R}(E) \defeq \coprod_{x \in X}\B_x
\]
Using local trivializations of the vector bundle $E$, we equip $\B_{\GL_k\R}(E)$
with the structure of a smooth manifold such that the map
$\pi : \B_{\GL_k\R}(E) \to X$ taking $\B_x$ to $x$ is a submersion. This gives
$\pi : \B_{GL_k\R} \to X$ the structure of a principal $\GL_k\R$-bundle, called
the \ib{frame bundle} of $E$,  where the local trivializations are defined in terms of
local trivializations of $E$.
\end{exmp}
%
\begin{exmp}
Let $E \to X$ be a rank $k$ vector bundle equipped with a fiber metric, i.e. a
smoothly  varying inner product on the fibers $E_x$. Then the
\ib{orthonormal frame bundle} of $E$, denoted $\B_{\mathrm{O}}(E)$, is the principal
$\mathrm{O}_k$-bundle where the fiber over $x \in X$ is the $\mathrm{O}_k$-torsor
of orthonormal bases for $E_x$.
\end{exmp}
%
A near identical story holds for complex vector bundles -- from any complex
vector bundle we get a principal $\mathrm{GL}_k\C$-bundle of frames, and if
we fix a Hermitian fiber metric, we get a principal $\mathrm{U}_k$-bundle of
orthonormal frames. \\

Principal bundles can be thought bundles of symmetries of some other fiber bundle,
which can be made precise using the notion of an associated bundle, which allows
one to construct fiber bundles out of principal bundles.
%
\begin{defn}
Let $P \to X$ be a principal $G$-bundle, and let $F$ be a smooth manifold with
right $G$ action. The \ib{associated fiber bundle}, denoted $P \times_G F$
(sometimes denoted $P \times^G F$) is the space
\[
P \times_G F \defeq (P \times F)/G
\]
where the right $G$-action on $F$ is the diagonal action, i.e.
$(p,f)\cdot g = (p\cdot g, f\cdot g)$.
\end{defn}
%
If instead we have a left $G$-action on $F$, we can turn it into a right action
by defining $f \cdot g \defeq g\inv \cdot f$. As the name suggests, $P \times_G F$ is
a fiber bundle.
%
\begin{exer}
Let $\pi : P \to X$ a principal bundle. Given a smooth right action of $G$ on
$F$, use local trivializations of $P \to X$ to show that the map taking an
equivalence class $[p,f]$ to $\pi(p)$ gives $P\times_G F$ the structure of a fiber
bundle over $X$ with model fiber $F$.
\end{exer}
%
In the case that the model fiber is a vector space $V$, and the action is
linear, the associated bundle $P \times_G V$ is a vector bundle.
%
\begin{exer} \enumbreak
\begin{enumerate}
  \item Let $E \to X$ be a rank $k$ vector bundle, and $\B_{\GL_k\R}(E)$ be its
  $\GL_k\R$-bundle of frames. Let $\rho : \GL_k\R \to \GL_k\R$ be the
  defining representation (i.e. the identity map). Show that the associated bundle
  $\B_{\GL_k\R}(E) \times_{\GL_k\R} \R^k$ is isomorphic to $E$.
  \item Further suppose that $E$ comes equipped with a fiber metric, and let
  $\B_{\mathrm{O}}(E)$  be its orthonormal frame bundle. The associated
  bundle $\B_{\mathrm{O}}(E) \times_{\mathrm{O}_k} \R^k$ is isomorphic to $E$
  by a near identical proof as the previous part. How can one recover the fiber
  metric?
  \item Let $\rho^* : \GL_k \to \GL_k$ denote the dual representation of the
  defining representation $\rho$. Show that the associated bundle
  is isomorphic to the dual bundle $E^*$. In particular, this should
  illuminate the distinction between the tangent and cotangent bundles.
\end{enumerate}
\end{exer}
%
Associated bundles have another nice feature -- their sections have
a nice interpretation in terms of $G$-equivariant maps.
%
\begin{prop}
Let $E = P \times_G F$ be an associated fiber bundle, and let
$\Gamma(X,E)$ denote the space of global sections, i.e. the space of
smooth maps $f : X \to E$ such that $\pi \circ f = \id_X$, where
$\pi : E \to X$ denotes the projection map. Then there is a bijective correspondence
\[
\Gamma(X,E) \longleftrightarrow \set{G-\text{equivariant maps } P \to F}
\]
\end{prop}
%
\begin{proof}
Let $\sigma : X \to E$ be a section. Then define the map
$\widetilde{\sigma} : P \to F$ as follows: for $x \in X$, let $(p,f)$ be a
representative for $\sigma(x)$. Then define $\widetilde{\sigma}(x) \defeq f$. \\

In the other direction, let $\widetilde{\varphi} : P \to F$ be an equivariant
map. Then define the section $\varphi : X \to E$ by
$\varphi(x) = [p,\widetilde{\varphi}(p)]$.
\end{proof}
%
\begin{exer}
Verify that the map $\varphi$ defined above is well-defined. Verify the two
constructions above are inverses to each other.
\end{exer}
%