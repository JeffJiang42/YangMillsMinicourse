%
\section{The Yang-Mills Equations}
%
To discuss the Yang-Mills equations, we will restrict to compact Lie groups
$G$. As before, $X$ will denote an $n$-dimensional closed smooth manifold. \\

Since $G$ is compact, its Lie algebra $\g$ is semisimple, so the Killing
form $\langle\cdot,\cdot\rangle : \g \otimes \g \to \R$ is nondegenerate.
For the rest of our discussion, $\langle\cdot,\cdot\rangle$ can be replaced by
any inner product invariant under the Adjoint action, though it does us no
harm to assume that it is the Killing form.
%
\begin{lem}
Let $\langle\cdot,\cdot\rangle$ denote any Adjoint invariant inner product on $\g$.
Then for $X_1,X_2,X_3 \in \g$, we have
\[
\langle[X_1,X_2],X_3\rangle = \langle X_1,[X_2,X_3]\rangle
\]
\end{lem}
%
\begin{proof}
We compute
\begin{align*}
\langle[X_1,X_2],X_3\rangle &= \langle[-X_2,X_1],X_3\rangle \\
&= \frac{d}{dt}\bigg\vert_{t=0}\langle\Ad_{\exp(-tX_2)}X_1,X_3\rangle \\
&= \frac{d}{dt}\bigg\vert_{t=0}
\langle\Ad_{\exp{tX_2}}\Ad_{\exp(-tX_2)}X_1,\Ad_{\exp(tX_2)}X_3\rangle \\
&= \langle X_1,[X_2,X_3]\rangle
\end{align*}
\end{proof}
%
The form $\langle\cdot,\cdot\rangle$ induces a fiber metric on
$P \times \g$, and invariance under the Adjoint action tells us that
this fiber metric descends to a fiber metric on $\g_P$. This gives
us pairings
\begin{align*}
\Omega^k_X(\g_P) \otimes \Omega^\ell_X(\g_P) &\to \Omega^{k+\ell}_X \\
\omega \otimes \eta &\mapsto \langle\omega,\eta\rangle
\end{align*}
We now fix an orientation and a Riemannian metric $g$ on $X$. This gives us:
\begin{enumerate}
  \item A Riemannian volume form $\mathrm{Vol}_g \in \Omega^n_X$.
  \item A Hodge star operator $\star : \Omega^k_X \to \Omega^{n-k}_X$.
  \item Fiber metrics $\langle\cdot,\cdot\rangle_g$ on the bundles
  $\Lambda^kT^*X$.
\end{enumerate}
%
The Hodge star extends to $\g_P$-valued forms, which gives us inner products
on $\Omega^k_X(\g_P)$ given by
\[
(\omega,\eta) \defeq \int_X \langle\omega,\star\eta\rangle
\]
We let $\norm{\cdot}$ denote the norm induced by these inner products. \\

We now introduce the gauge group of a prinicipal $G$-bundle $P \to X$.
%
\begin{defn}
Let $\pi : P \to X$ be a principal $G$-bundle. The \ib{gauge group}, denoted
$\mathscr{G}(P)$, is the group of automorphisms of $P$, i.e. $G$-equivariant
diffeomorphisms $\varphi : P \to P$ such that $\pi = \pi \circ \varphi$.
An element of $\mathscr{G}(P)$ is called a \ib{gauge transformation}.
\end{defn}
%
\begin{prop}
The group $\mathscr{G}(P)$ is isomorphic to the group of sections
$\Gamma(X,\Ad P)$, where the group operation is pointwise multiplication.
\end{prop}
%
\begin{proof}
We provide maps in both directions. Suppose we have an automorphism
$\varphi : P \to P$. Since $\pi = \pi \circ \varphi$, the map $\varphi$ preserves
the fibers of $\pi$. Therefore, for any $p \in P$, we have that
$p$ and $\varphi(p)$ differ by the action of some $g_p \in G$. The mapping
$g_\varphi : P \to G$ taking $p \mapsto g_p$ is easilty verified to be
equivariant with respect to the conjugation action of $G$, so it defines
a section of $\Ad P$ \\

In the other direction, given a $G$-equivariant map
$f : P \to G$, we get a bundle automorphism $\varphi_f : P \to P$
where  $\varphi_f(p) = p \cdot f(p)$. The two maps we constructed are clearly inverse
to each other, giving us the desired correspondence.
\end{proof}
%
The gauge group $\mathscr{G}(P)$ acts on the space $\Omega^1_P(\g)$ of
$\g$-valued forms by pullback. We claim that it preserves the
subspace $\mathscr{A}(P) \subset \Omega^1_P(\g)$.
%
\begin{prop}
For a connection $A \in \mathscr{A}(P)$ and a gauge transformation
$\varphi : P \to P$, we have
\begin{enumerate}
  \item $R_g^*A = \Ad_{g\inv}\varphi^*A$.
  \item $\iota_X\varphi^*A = X$ for all $X \in \g$.
\end{enumerate}
Equivalently, if we let $g_\varphi : P \to G$ denote the equivariant map
associated to $\varphi$, we have
\[
\varphi^*A = \Ad_{g\inv_\varphi A} + g_\varphi^*\theta
\]
where $\theta \in \Omega^1_G(\g)$ is the Maurer-Cartan form.
\end{prop}
%
\begin{exer}
Prove the previous proposition.
\end{exer}
%
\begin{defn}
Two connections $A_1$ and $A_2$ are \ib{gauge equivalent} if there exists
a gauge transformation $\varphi \in \mathscr{G}(P)$ such that $\varphi^*A_1 = A_2$.
\end{defn}
%
\begin{prop}
Let $A \in \mathscr{A}(P)$ be a connection, $\varphi : P \to P$ a gauge
transformation, and $g_\varphi : P \to G$ the associated equivariant map.
Then
\[
F_{\varphi^*A} = \Ad_{g_\varphi\inv}F_A
\]
\end{prop}
%
\begin{proof}
We compute
\begin{align*}
&F_{\varphi^*A} = d(\Ad_{g_\varphi\inv} A + g_\varphi^*\theta)
+ \frac{1}{2}[\Ad_{g_\varphi\inv} A + g_\varphi^*\theta\wedge
\Ad_{g_\varphi\inv} A + g_\varphi^*\theta] \\
&= \Ad_{g_\varphi\inv}dA + g_\varphi^*d\theta + \frac{1}{2}
\left([\Ad_{g_\varphi\inv}A\wedge \Ad_{g_\varphi\inv} A] + [\Ad_{g_\varphi\inv} A\wedge
g_\varphi^*\theta] + [g^*_\varphi\theta\wedge \Ad_{g_\varphi\inv}A]
+ [g_\varphi^*\theta\wedge g_\varphi^*\theta] \right) \\
&= \Ad_{g_\varphi\inv} dA + \frac{1}{2}[\Ad_{g_\varphi\inv}A, \Ad_{g_\varphi\inv} A]] \\
&= \Ad_{g_\varphi\inv F_A}
\end{align*}
%
The term
\[
g_\varphi^*d\theta + \frac{1}{2}[g_\varphi^*\theta\wedge g_\varphi^*\theta]
\]
vanishes due to the Maurer-Cartan equation. The term
\[
\frac{1}{2}\left([\Ad_{g_\varphi\inv} A\wedge
g_\varphi^*\theta] + [g^*_\varphi\theta\wedge \Ad_{g_\varphi\inv}A]\right)
\]
vanishes due to the fact that $[\cdot,\cdot]$ is skew symmetric on $1$-forms.
\end{proof}
%
With some of the preliminary results established, we arrive at the Yang-Mills
functional.
%
\begin{defn}
The \ib{Yang-Mills functional} is the map $L : \mathscr{A}(P) \to \R$ given by
\[
L(A) \defeq \norm{F_A}^2 = \int_X \langle F_A\wedge\star F_A\rangle
\]
\end{defn}
%
We note that for any gauge transformation $\varphi \in \mathscr{G}(P)$,
we have $L(\varphi^*A) = L(A)$, since we have
\[
L(\varphi^*A)
= \int_X\langle \Ad_{g_\varphi\inv}F_A\wedge\star\Ad_{g_\varphi\inv}F_A\rangle
= \int_X\langle F_A\wedge\star F_A\rangle = L(A)
\]
because of this we say that $L$ is \ib{gauge invariant}. \\

The Yang-Mills equations are the variational equations for the Yang-Mills
functional.
%
\begin{prop}[\ib{The first variation}]
Let $A$ be a local extremum of $L$. Then we have
\[
d_A\star F_A = 0
\]
\end{prop}
%
\begin{proof}
Let $\eta \in \Omega^1_X(\g_P)$. We then compute
\begin{align*}
L(A) &= \int_X \langle F_{A+t\eta}\wedge\star F_{A+t\eta} \\
&= \int_X\langle F_A + \frac{t^2}{2}[\eta\wedge\eta] + td_A\eta\wedge
\star(F_A + \frac{t^2}{2}[\eta\wedge\eta] + td_A\eta)
\end{align*}
The term linear in $t$ is
\[
\int_X \langle F_A\wedge\star d_A\eta + \langle d_A\eta\wedge\star F_A\rangle =
2(F_A,d_A\eta)
\]
Then let $d^*_A = (-1)^{2n+1}\star d_A\star$ denote the formal adjoint to
$d_A$. Since $A$ is a local extremum, the term linear in $t$ must vanish,
so for every $\eta$, we must have
\[
(F_A,d_A\eta) = (d^*_AF_A,\eta) = 0
\]
Then since up to sign $d^*_A = \star d_A \star$ and $\star$ is an isomorphism,
we have $d_A\star F_A = 0$.
\end{proof}
%
The first variation gives us what are referred to as the \ib{Yang-Mills equations}
\begin{align*}
d_A F_A &= 0 \\
d_A^*F_A &= 0
\end{align*}
%
\begin{defn}
A \ib{Yang-Mills connection} is a connection $A \in \mathscr{A}(P)$ satisfying
the Yang-Mills equations, i.e. a local extremum of $L$.
\end{defn}
%
\begin{exer}
In the case that $G = \mathrm{U}_1$, show that the curvature of a connection
$A$ can be identified as an element of $\Omega^2_X$. Show that $A$ is a
Yang-Mills connection if and only if $F_A$ is a harmonic form, i.e.
$\Delta F_A = 0$, where $\Delta = dd^* + d^*d$ is the Hodge Laplacian.
Use this to show that the space of Yang-Mills connections on a principal
$\mathrm{U}_1$-bundle $P$ is a torsor over the vector space of closed $1$-forms
on $X$.
\end{exer}
%
The first equation is simply the \ib{Bianchi identity} and the second comes
from the first variation.
%
\begin{prop}[\ib{The second variation}]
Let $A$ be a Yang-Mills connection. Then for every $\eta \in \Omega^1_X(\g_P)$, we have
\[
\frac{d}{dt}\bigg\vert_{t=0}d^*_{A+t\eta}F_{A+t\eta} =
d^*_Ad_A\eta + \star[\eta\wedge\star F_A]
\]
\end{prop}
%
The proof of this is similar to the proof of the first variation,
and involves expanding out $d^*_{A+t\eta}F_{A+t\eta}$ and then taking
the term linear in $t$. If we think of $L$ as a Morse function on
$\mathscr{A}(P)$, for a Yang-Mills connection $A$, the operator
$d^*_Ad_A + \star[\cdot\wedge\star F_A]$ can be interpreted as the Hessian
of $L$ at the critical point $A$. In particular, if $\eta$ is
tangent to the critical submanifold of Yang-Mills connections,
one can use the Atiyah-Singer index theorem with this operator to
compute the dimension of the space of Yang-Mills connections. \\

We now restrict ourselves to the case where $X$ is a Riemann surface. Let
$\Gamma_\R$ denote the central extension of $\pi_1(X)$ by
$\R$ where if we let $J$ denote a generator for $\R$, we have the
relation $\prod_{i}[a_i,b_i] = J$ where the $a_i$ and $b_i$ are the generators
for the usual presentation of a closed surface of genus $g$. Using this
group one can prove the following theorems, though we will omit the proofs.
%
\begin{thm}
Every principal $G$-bundle $P \to X$ admits a Yang-Mills connection.
\end{thm}
%
\begin{thm}
There is a bijective correspondence
\[
\hom(\Gamma_\R,G)/G \longleftrightarrow\set{\text{Principal }G\text{-bundles }
P \to X \text{with a Yang-Mills connection}}/\sim
\]
where the action of $G$ is conjugation and the equivalence relation is
gauge equivalence.
\end{thm}
%
The second theorem should be thought of an analogue of the classical
Riemann-Hilbert correspondence.
%
\begin{exer}
The classical Riemann-Hilbert correspondence gives a bijection
\[
\hom(\pi_1(X),G)/G \leftrightarrow \set{\text{Principal }G\text{-bundles } P \to X
\text{ equipped with a flat connection } A}/\sim
\]
where the action of $G$ on $\hom(\pi_1(X),G)$ is by conjugation, and
the equivalence relation is gauge equivalence of conenctions.
The correspondence assigns to $\rho \in \hom(\pi_1(X),G)$ the associated bundle
\[
\widetilde{X} \times_{\pi_1(X)} G
\]
where $\widetilde{X}$ is the universal cover of $X$, and the connection
is the one induced by descending the trivial connection on $\widetilde{X} \times G$
to the quotient. In the other direction, the holonomy of a flat connection
defines (up to conjugation by $G$) a homomorphism $\pi_1(X) \to G$. \\

A principal $\mathrm{U}_1$-bundle $P \to X$ corresponds to a Hermitian line bundle
$L \to X$ by taking the associated bundle $P\times_{\mathrm{U}_1}\C$ with
the standard action of $\mathrm{U}_1$ on $\C$. Using this correspondence
and the classical Riemann-Hilbert correspondence, show that there is a bijection
\[
\hom(\Gamma_\R,\mathrm{U}_1)/\mathrm{U}_1
\longleftrightarrow\set{\text{Principal }\mathrm{U}_1\text{-bundles }
P \to X \text{with a Yang-Mills connection}}/\sim
\]
\end{exer}
%
For the rest of our discussion, we will restrict to case where
$G = \mathrm{U}_n$. We first make a remark involving the proofs of the
two preceding theorems. As with the $\mathrm{U}_1$ case, the data of
a principal $\mathrm{U}_n$-bundle $P \to X$ is equivalent to a
rank $n$ complex vector bundle $E \to X$ equipped with a Hermitian metric.
In the proofs, one shows that a Yang-Mills connection $A$ is equivalent to the
choice of a Lie algebra element $X \in \mathfrak{u_n}$. Writing
$X = -2\pi i \Lambda$ for a Hermitian matrix $\Lambda$, the Yang-Mills
condition implies that the trace of $\Lambda$ is equal to the first Chern
class of $E$, thought of as an integer by integrating over $X$. If
we let $\lambda_i$ denote the $i^{th}$-eigenvalue (arranged in ascending order)
and $n_i$ the multiplicity of $\lambda_i$, one can show that $n_i\lambda_i$ must
also be integral. These observations will be useful when we relate
Yang-Mills connections with holomorphic vector bundles.
%